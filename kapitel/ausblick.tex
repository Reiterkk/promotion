Zukünftige Implementierungsarbeiten könnten die Berechnung von Abschirmungskonstanten für paramagnetische Systeme ermöglichen, wofür der offenschalige Formalismus auf die Gleichungen zur Berechnung von \ac{nmr}-Verschiebungen übertragen und implementiert werden muss. Mit der Implementierung der \acp{ecp} ist es gelungen, die Berechnung für Moleküle mit schweren Elementen zu ermöglichen. Jedoch umfasst das Modul \texttt{mpshift} bislang nicht die Möglichkeit, Abschirmungskonstanten für diese schweren Elemente selbst zu berechnen. Aus diesem Grund ist die Implementierung einer relativistischen All-Elektronen-Methode von großem Interesse. Neben den eindimensionalen \ac{nmr}-Spektren ist auch eine Implementierung zur Berechnung von zweidimensionalen Spektren denkbar.  

Durch die in der vorliegenden Arbeit implementierten Methoden zur Verbesserung der Effizienz bei der Berechnung von Abschirmungskonstanten wird die Berechnung des exakten Austauschs zum zeitbestimmenden Schritt für Hartree-Fock- oder Hybrid-\ac{dft}-Rechnungen. Für kleine Basissätze von \textit{double}-$\zeta$- oder \textit{triple}-$\zeta$-Qualität wurde bereits eine effizientere Integralabschätzung implementiert. Wird zu größeren Basissätzen übergegangen, besteht wie beim Coulombbeitrag die Möglichkeit, Näherungsverfahren zu verwenden. Im \textsc{Turbomole}-Programmpaket sind für die Berechnung der Wellenfunktion   bereits die \ac{rik}-Näherung\supercite{weigend2002fully} und eine semi-numerische Methode\supercite{plessow2012seminumerical} zur Berechnung des Austauschs implementiert. Die Erweiterung dieser Techniken zur Berechnung der nach den Komponenten des Magnetfeldes abgeleiteten Austauschintegrale ist im Anhang \ref{hfaustausch} skizziert. 
\vfill
\newpage
\thispagestyle{empty}
\cleardoublepage


