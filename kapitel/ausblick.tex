Zukünftige Implementierungsarbeiten könnten die Berechnung von chemischen Abschirmungskonstanten für paramagnetische Systeme ermöglichen, wofür der offenschalige Formalismus auf die Gleichungen zur Berechnung von \ac{nmr}-Verschiebungen übertragen und implementiert werden muss. Mit der Implementierung der \acp{ecp} ist es gelungen, die Berechnung für Moleküle mit schweren Elementen zu ermöglichen. Jedoch umfasst das Modul \texttt{mpshift} bislang nicht die Möglichkeit, chemische Abschirmungskonstanten für diese schweren Elemente selbst zu berechnen. Aus diesem Grund ist die Implementierung einer relativistischen All-Elektronen-Methode von großem Interesse. Neben den eindimensionalen \ac{nmr}-Spektren ist auch eine Implementierung zur Berechnung von zweidimensionalen Spektren denkbar.  
Durch die in der vorliegenden Arbeit implementierten Methoden zur Verbesserung der Effizienz bei der Berechnung von chemischen Abschirmungskonstanten wird die Berechnung des exakten Austauschs zum zeitbestimmenden Schritt für Hartree-Fock- oder Hybrid-\ac{dft}-Rechnungen. Für kleine Basissätze von \textit{double}-$\zeta$- oder \textit{triple}-$\zeta$-Qualität wurde bereits eine effizientere Integralabschätzung implementiert. Wird zu größeren Basissätzen übergegangen, besteht analog zum Coulombbeitrag die Möglichkeit, Näherungsverfahren zu verwenden. Bei der Berechnung der Wellenfunktion sind im \textsc{Turbomole}-Programmpaket bereits die \ac{rik}-Näherung\supercite{weigend2002fully} und eine semi-numerische Methode\supercite{plessow2012seminumerical} zur Berechnung des Austauschs implementiert. Diese Methoden lassen sich auch auf die Berechnung von \ac{nmr}-Verschiebungen übertragen. Die wichtigsten Gleichungen dafür werden in den beiden folgenden Abschnitten hergeleitet. 

\section{Semi-numerischer Austausch}
Die grundsätzliche Idee hinter der semi-numerischen Berechnung von Vierzentren-Zweielektronen-Integralen ist, die Integration der einen Elektronenkoordinate analytisch und die der anderen numerisch durchzuführen. Dafür wird die analytische Auswertung der Zweielektronen-Integrale durch die analytische Berechnung von Einelektronen-Integralen, gefolgt von einer numerischen Integration auf einem Gitter, ersetzt.\supercite{plessow2012seminumerical} Für die Austauschmatrixelemente $K_{\mu\nu}$ ergibt sich durch diesen Ansatz
%	\begin{equation}
%	K_{\mu\nu\kappa\lambda}=\left(\chi_\mu\chi_\lambda\vert\chi_\kappa\chi_\nu\right)\approx \sum_{g}^{N_{\textrm{g}}} X_{\mu g}X_{\lambda g}A_{\kappa\nu g}
%	\end{equation}		
	\begin{equation}\label{eq:senex}
%	\begin{aligned}
	K_{\mu\nu}=\frac{1}{2}\sum_{\kappa\lambda}^{N_{\textrm{BF}}}D_{\kappa\lambda} \left(\chi_\mu\chi_\lambda\vert\chi_\kappa\chi_\nu\right)
	\approx\frac{1}{2}\sum_{g}^{N_{\textrm{Git}}}X_{\mu g}\underbrace{\sum_\kappa^{N_{\textrm{BF}}}A_{\kappa\nu g}\underbrace{\sum_\lambda^{N_{\textrm{BF}}}D_{\kappa\lambda}X_{\lambda g}}_{F_{\kappa g}}}_{G_{\nu g}}
%	\end{aligned}
	\end{equation}
	mit 
	
	\begin{equation}
	X_{\mu g}=w_g^{\frac{1}{2}}\chi_\mu(\vec{r}_g)
	\end{equation}
	
	und
	\begin{equation}
	A_{\kappa\nu g}=A_{\kappa\nu}(\vec{r}_g)=\int \frac{1}{\vert\vec{r}_g-\vec{r}_2\vert}\chi_\kappa(\vec{r}_2)\chi_\nu(\vec{r}_2)\Diff3\vec{r}_2\, .
	\end{equation}
	
Die $X_{\mu g}$ sind die Werte der Basisfunktionen, multipliziert mit einem Gewichtungsfaktor $w_g^{\frac{1}{2}}$, und die $A_{\kappa\nu g}$ sind die analytisch ausgewerteten Einelektronen-Integrale auf den $N_{\textrm{Git}}$ Gitterpunkten $g$. Für die Berechnung von chemischen Abschirmungskonstanten wird die Ableitung von $K_{\mu\nu}$ nach den Komponenten des Magnetfeldes benötigt,
	\begin{equation}\label{eq:senexkmunudb}
	\begin{aligned}
	\left.\frac{\partial K_{\mu\nu}}{\partial B_\beta}\right\vert_{\vec{B}=0}=&K_{\mu\nu}^{B_\beta}\\
	=&\underbrace{\frac{1}{2}\sum_{\kappa\lambda}^{N_{\textrm{BF}}}D_{\kappa\lambda}^{B_\beta} \left(\chi_\mu^{\vec{B}=0}\chi_\lambda^{\vec{B}=0}\vert\chi_\kappa^{\vec{B}=0}\chi_\nu^{\vec{B}=0}\right)}_{=K_{\mu\nu}^{B_\beta}[D^{B_\beta}]}\\
	&\underbrace{+\frac{1}{2}\sum_{\kappa\lambda}^{N_{\textrm{BF}}}D_{\kappa\lambda} \underbrace{\left[\left(\overline{\chi_\mu\chi_\lambda}\vert\chi_\kappa^{\vec{B}=0}\chi_\nu^{\vec{B}=0}\right)_\beta+\left(\chi_\mu^{\vec{B}=0}\chi_\lambda^{\vec{B}=0}\vert\overline{\chi_\kappa\chi_\nu}\right)_\beta\right]}_{=K_{\mu\nu\kappa\lambda}^{B_\beta}}}_{=K_{\mu\nu}^{B_\beta}[D]}\, .
	\end{aligned}
	\end{equation}

Der erste Term auf der rechten Seite von Gleichung (\ref{eq:senexkmunudb}), $K_{\mu\nu}^{B_\beta}[D^{B_\beta}]$,  beinhaltet die gestörte Dichtematrix und die ungestörten Austauschintegrale. Dieser wird für Hartree-Fock- und Hybrid-\ac{dft}-Rechnungen während der \ac{cphf}-Iterationen benötigt. Die Implementierung dieses Terms ist trivial und wurde bereits in Zusammenarbeit mit Robert Treß im Rahmen seiner Masterarbeit umgesetzt. Es müssen lediglich die drei Komponenten der gestörten Dichtematrix an die entsprechenden Routinen übergeben und auf die Antisymmetrie der Matrizen geachtet werden. Die abgeleiteten Vierzentren-Zweielektronen-Integrale im zweiten Term auf der rechten Seite von Gleichung (\ref{eq:senexkmunudb}) sind
	\begin{equation}
	\begin{aligned}
	\left(\overline{\chi_\mu\chi_\lambda}\vert\chi_\kappa^{\vec{B}=0}\chi_\nu^{\vec{B}=0}\right)_\beta=&\frac{\iu}{2c} \left(\left(\vec{R}_{\mu\lambda}\times\vec{r}_1\right)_\beta\chi_\mu^{\vec{B}=0}\chi_\lambda^{\vec{B}=0}\vert\chi_\kappa^{\vec{B}=0}\chi_\nu^{\vec{B}=0}\right)\\
	=&\frac{\iu}{2c}\left(\vec{R}_\mu\times\vec{R}_\lambda\right)_\beta\left(\chi_\mu^{\vec{B}=0}\chi_\lambda^{\vec{B}=0}\vert\chi_\kappa^{\vec{B}=0}\chi_\nu^{\vec{B}=0}\right)\\
	&+\frac{\iu}{2c}\left[\vec{R}_{\mu\lambda}\times\left(\vec{r}_\mu\chi_\mu^{\vec{B}=0}\chi_\lambda^{\vec{B}=0}\vert\chi_\kappa^{\vec{B}=0}\chi_\nu^{\vec{B}=0}\right)\right]_\beta,
	\end{aligned}
	\end{equation}
	
	\begin{equation}
	\begin{aligned}
	\left(\chi_\mu^{\vec{B}=0}\chi_\lambda^{\vec{B}=0}\vert\overline{\chi_\kappa\chi_\nu}\right)_\beta=&\frac{\iu}{2c}\left(\vec{R}_\kappa\times\vec{R}_\nu\right)_\beta\left(\chi_\mu^{\vec{B}=0}\chi_\lambda^{\vec{B}=0}\vert\chi_\kappa^{\vec{B}=0}\chi_\nu^{\vec{B}=0}\right)\\
	&+\frac{\iu}{2c}\left[\vec{R}_{\kappa\nu}\times\left(\chi_\mu^{\vec{B}=0}\chi_\lambda^{\vec{B}=0}\vert\chi_\kappa^{\vec{B}=0}\vec{r}_\nu\chi_\nu^{\vec{B}=0}\right)\right]_\beta.
	\end{aligned}
	\end{equation}

Exemplarisch ergibt sich für die Ableitung nach der $x$-Komponente 	
	\begin{equation}
	\begin{aligned}
	K_{\mu\nu\kappa\lambda}^{B_x}=&\frac{\iu}{2c}\left[(R_{\mu\nu\kappa\lambda})_x\left(\chi_\mu^{\vec{B}=0}\chi_\lambda^{\vec{B}=0}\vert\chi_\kappa^{\vec{B}=0}\chi_\nu^{\vec{B}=0}\right)\right.\\
	&+\left(R_{\mu\lambda}\right)_y\left(z_\mu\chi_\mu^{\vec{B}=0}\chi_\lambda^{\vec{B}=0}\vert\chi_\kappa^{\vec{B}=0}\chi_\nu^{\vec{B}=0}\right)-\left(R_{\mu\lambda}\right)_z\left(y_\mu\chi_\mu^{\vec{B}=0}\chi_\lambda^{\vec{B}=0}\vert\chi_\kappa^{\vec{B}=0}\chi_\nu^{\vec{B}=0}\right)\\
	&\left.+\left(R_{\kappa\nu}\right)_y\left(\chi_\mu^{\vec{B}=0}\chi_\lambda^{\vec{B}=0}\vert\chi_\kappa^{\vec{B}=0}z_\nu\chi_\nu^{\vec{B}=0}\right)-\left(R_{\kappa\nu}\right)_z\left(\chi_\mu^{\vec{B}=0}\chi_\lambda^{\vec{B}=0}\vert\chi_\kappa^{\vec{B}=0}y_\nu\chi_\nu^{\vec{B}=0}\right)\right]
	\end{aligned}
	\end{equation}
	
	mit
	\begin{equation}
	(R_{\mu\nu\kappa\lambda})_x=(R_\mu)_y(R_\lambda)_z-(R_\mu)_z(R_\lambda)_y+(R_\kappa)_y(R_\nu)_z-(R_\kappa)_z(R_\nu)_y.
	\end{equation}	
	
	Die Übertragung des semi-numerischen Ansatzes aus Gleichung (\ref{eq:senex}) liefert schließlich
	\begin{equation}
	\begin{aligned}
	K_{\mu\nu}^{B_\beta}[D]\approx\frac{1}{2}\sum_{g}^{N_{\textrm{GIT}}}&\left[X_{\mu g}(G_{\mu\nu g})_\beta^0-\left(\vec{X}_{\mu g}^+\times\vec{G}_{\mu\nu g}^{\,1}\right)_\beta-\left(\vec{X}_{\nu g}^+\times\vec{G}_{\mu\nu g}^{\,2}\right)_\beta\right]
	\end{aligned}
	\end{equation}
	
	mit den darin auftretenden Größen
	\begin{equation}
	\begin{aligned}
	&(G_{\mu\nu g})_\alpha^0=\sum_\kappa^{N_{\textrm{BF}}}A_{\kappa\nu g}(F_{\mu\nu\kappa g})_\alpha^0\, ,\\
	&(G_{\mu\nu g})_\alpha^1=\sum_\kappa^{N_{\textrm{BF}}}A_{\kappa\nu g}(F_{\mu\kappa g})_\alpha^1\, ,\\
	&(G_{\mu\nu g})_\alpha^2=\sum_\lambda^{N_{\textrm{BF}}}A_{\lambda\mu g}(F_{\nu\lambda g})_\alpha^2\, ,
	\end{aligned}
	\end{equation}	
	
	\begin{equation}
	\begin{aligned}
	&(F_{\mu\nu\kappa g})_\alpha^0=\sum_\lambda^{N_{\textrm{BF}}}D_{\kappa\lambda}X_{\lambda g}(R_{\mu\nu\kappa\lambda})_\alpha\, ,\\
	&(F_{\mu\kappa g})_\alpha^1=\sum_\lambda^{N_{\textrm{BF}}}D_{\kappa\lambda}X_{\lambda g}(R_{\mu\lambda})_\alpha\, ,\\
	&(F_{\nu\lambda g})_\alpha^2=\sum_\kappa^{N_{\textrm{BF}}}D_{\kappa\lambda}X_{\kappa g}(R_{\kappa\nu})_\alpha
	\end{aligned}
	\end{equation}
	
	und
	\begin{equation}
	\left(X_{\mu g}^+\right)_\alpha=w_g^{\frac{1}{2}}\alpha_\mu\chi_\mu(\vec{r}_g)\, .
	\end{equation}
	
	Eingesetzt ergibt sich für die $x$-Komponente
	\begin{equation}
	\begin{aligned}
	K_{\mu\nu}^{B_x}[D]\approx\frac{1}{2}\sum_{g}^{N_{\textrm{GIT}}}&\left[X_{\mu g}\sum_\kappa^{N_{\textrm{BF}}}A_{\kappa\nu g}\sum_\lambda^{N_{\textrm{BF}}}D_{\kappa\lambda}X_{\lambda g}(R_{\mu\nu\kappa\lambda})_x\right.\\
	+&\left(X_{\mu g}\right)_z\sum_\kappa^{N_{\textrm{BF}}}A_{\kappa\nu g}\sum_\lambda^{N_{\textrm{BF}}}D_{\kappa\lambda}X_{\lambda g}(R_{\mu\lambda})_y\\
	-&\left(X_{\mu g}\right)_y\sum_\kappa^{N_{\textrm{BF}}}A_{\kappa\nu g}\sum_\lambda^{N_{\textrm{BF}}}D_{\kappa\lambda}X_{\lambda g}(R_{\mu\lambda})_z\\
	+&\left(X_{\nu g}\right)_z\sum_\lambda^{N_{\textrm{BF}}}A_{\lambda\mu g}\sum_\kappa^{N_{\textrm{BF}}}D_{\kappa\lambda}X_{\kappa g}(R_{\kappa\nu})_y\\
	-&\left.\left(X_{\nu g}\right)_y\sum_\lambda^{N_{\textrm{BF}}}A_{\lambda\mu g}\sum_\kappa^{N_{\textrm{BF}}}D_{\kappa\lambda}X_{\kappa g}(R_{\kappa\nu})_z\right].
	\end{aligned}
	\end{equation}

\section{\acs{ri}-Methode für den Hartree-Fock-Austausch}
Neben der semi-numerischen Auswertung der Austauschintegrale kann auch das \ac{rik}-Verfahren auf die Berechnung der Vierzentren-Zweielektronen-Integrale angewendet werden.\supercite{weigend2002fully} Diese Näherung lässt sich auf die nach den Komponenten des Magnetfeldes abgeleitete Austauschmatrix übertragen. Das Einsetzen der \ac{rik}-Näherung in den zweiten Term auf der rechten Seite von Gleichung (\ref{eq:senexkmunudb}) liefert
\begin{equation}
\begin{aligned}
K_{\mu\nu}^{B_\beta}[D]=&\frac{1}{2}\sum_{\kappa\lambda}^{N_{\textrm{BF}}}D_{\kappa\lambda} \left[\left(\overline{\chi_\mu\chi_\lambda}\vert\chi_\kappa^{\vec{B}=0}\chi_\nu^{\vec{B}=0}\right)_\beta+\left(\chi_\mu^{\vec{B}=0}\chi_\lambda^{\vec{B}=0}\vert\overline{\chi_\kappa\chi_\nu}\right)_\beta\right]\\
\approx& \sum_{PQR}^{N_{\text{AuxBF}}}\sum_i^{N/2}\sum_{\kappa\lambda}^{N_{\textrm{BF}}}\left[ c_{\lambda i} \left(\overline{\chi_\mu\chi_\lambda}\vert P\right)_\beta\left(P\vert Q\right)^{-\frac{1}{2}} \left(Q\vert R\right)^{-\frac{1}{2}} \left(R \vert\chi_\kappa^{\vec{B}=0}\chi_\nu^{\vec{B}=0}\right)c_{\kappa i} \right.\\
&\left.+c_{\lambda i}\left(\chi_\mu^{\vec{B}=0}\chi_\lambda^{\vec{B}=0}\vert P\right)\left(P\vert Q\right)^{-\frac{1}{2}} \left(Q\vert R\right)^{-\frac{1}{2}} \left(R\vert\overline{\chi_\kappa\chi_\nu}\right)_\beta c_{\kappa i}\right].
\end{aligned}
\end{equation}

Analog zum Vorgehen bei der Berechnung der Wellenfunktion werden zunächst die intermediären Größen 
\begin{equation}
\overline{B}_{i\mu}^{\,QB_\beta}=\sum_P^{N_{\text{AuxBF}}}\sum_{\lambda}^{N_{\textrm{BF}}}c_{\lambda i} \left(\overline{\chi_\mu\chi_\lambda}\vert P\right)_\beta\left(P\vert Q\right)^{-\frac{1}{2}}\, ,
\end{equation}
\begin{equation}
B_{i\nu}^{\,Q}=\sum_R^{N_{\text{AuxBF}}}\sum_{\kappa}^{N_{\textrm{BF}}}\left(Q\vert R\right)^{-\frac{1}{2}} \left(R \vert\chi_\kappa^{\vec{B}=0}\chi_\nu^{\vec{B}=0}\right)c_{\kappa i}
\end{equation}

ausgewertet. Aus diesen lassen sich in einem letzten Schritt die nach den Komponenten des Magnetfeldes abgeleiteten Austauschmatrixelemente berechnen,
\begin{equation}
K_{\mu\nu}^{B_\beta}[D]=\sum_Q^{N_{\text{AuxBF}}}\sum_i^{N/2}\left(\overline{B}_{i\mu}^{\,QB_\beta} B_{i\nu}^{\,Q}+B_{i\mu}^{\,Q}\overline{B}_{i\nu}^{\,QB_\beta}\right).
\end{equation}