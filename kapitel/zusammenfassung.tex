Im Rahmen der vorliegenden Arbeit konnte die Funktionalität des Moduls \texttt{mpshift} deutlich erweitert und die Effizienz bei den Berechnungen maßgeblich gesteigert werden. Durch die Implementierung des \acf{cosmo} besteht nun die Möglichkeit, Umgebungseffekte bei der Berechnung von Abschirmungskonstanten einzubeziehen. Zusätzlich ermöglicht dieses Modell die Kompensation von Gegenionen, was insbesondere für hoch geladene anionische Verbindungen wichtig ist. Darüber hinaus wurde das \acf{dcosmors} implementiert und in einer ausführlichen Studie mit dem \ac{cosmo} verglichen. Es konnte jedoch kein klarer Favorit innerhalb der beiden Methoden ausgemacht werden.

\bigskip
Die Implementierung von \acfp{ecp} ermöglicht nun die Berechnung von Abschirmungskonstanten in Molekülen, welche schwere Elemente beinhalten, ohne große All-Elektronen-Basissätze verwenden zu müssen. Auf diese Weise wird zusätzlich die Auswirkung relativistischer Effekte auf die Abschirmungskonstanten der Nachbaratome schwerer Elemente berücksichtigt. Die Eichinvarianz der Implementierung wurde anhand der Abschirmungskonstanten in Co(ppy)$_3$ (ppy=2-Phenylpyridin) demonstriert. 

\bigskip
Das Modul \texttt{mpshift} kann nun ebenfalls die Response der Wellenfunktion auf ein äußeres Magnetfeld, welche zur Berechnung von \ac{vcd}-Spektren benötigt wird, bereitstellen. Die eigentliche Berechnung der \ac{vcd}-Spektren wurde in das im Modul \texttt{aoforce} implementiert. Der Vergleich zwischen dem experimentellen \ac{vcd}-Spektrum und den simulierten \ac{vcd}-Spektren von \acl{cpa} zeigt, dass die effiziente Basissatz/Funktional-Kombination def2-SV(P)/BP86 qualitativ gleichwertige Spektren liefert wie die deutlich aufwändigere Kombination def2-TZVP/B3LYP. Die effiziente Implementierung aus der vorliegenden Arbeit erlaubt beispielsweise die Berechnung der \ac{vcd}-Spektren von großen $I$-symmetrischen Fullerenen. Darunter ist das C$_{620}^{2+}$ mit 8680 Basisfunktionen das bislang größte System, für welches ein \ac{vcd}-Spektrum berechnet wurde.

Des Weiteren wurde das Modul \texttt{mpshift} in Zusammenarbeit mit Fabian Mack im Rahmen seiner Masterarbeit um die Möglichkeit erweitert, Abschirmungskonstanten mithilfe von \ac{mgga}-Funktionalen zu berechnen. 
\vfill
\newpage
Durch die Implementierung der \ac{ri}- und \ac{marij}-Näherungen konnte eine hoch effiziente Berechnung des Coulombbeitrages erreicht werden. Dies führt beispielsweise bei der Berechnung der Abschirmungskonstanten eines \ac{rns}-Segments mit 1018 Atomen und 10220 Basisfunktionen zu einer Rechenzeitbeschleunigung des Coulombbeitrages um einen Faktor größer als 100, ohne einen nennenswerten Verlust an Genauigkeit. Die Gesamtrechenzeit reduziert sich damit von \unit[43.2]{h} auf \unit[7.6]{h}. Neben weiteren Optimierungen des Programmcodes konnte die Anzahl der benötigten Iterationen beim Lösen der \ac{cphf}-Gleichungen leicht verringert werden. Für Hartree-Fock- bzw. Hybrid-\ac{dft}-Rechnungen wurde zudem eine effizientere Integralabschätzung auf die Berechnung der nach den Komponenten des Magnetfeldes abgeleiteten Austauschmatrixelemente übertragen. Die Berechnung der Abschirmungskonstanten für eine Kette von 48 $\alpha$-\textsc{d}-Glucose-Einheiten mit dem B3LYP-Funktional benötigt mit der vorliegenden Implementierung \unit[5.75]{h} auf einer einzelnen CPU. Dies ist ein Bruchteil der für andere Implementierungen (Referenzen \cite{beer2011nuclei} und \cite{kumar2016nuclei}) publizierten Zeiten für die identischen Rechnungen. Mit der vorliegenden Implementierung lassen sich die Abschirmungskonstanten von Systemen mit mehreren tausend Atomen innerhalb weniger Tage auf (Hybrid-)\ac{dft}-Niveau unter Verwendung einer \textit{double}-$\zeta$-Basis berechnen. Der Aufwand für die Berechnung der chemischen Verschiebungen liegt nun oft unter dem für die vorangehende Berechnung der Wellenfunktion. Zusätzlich wurden alle zeitbestimmenden Routinen im Modul \texttt{mpshift} parallelisiert, um die Rechnungen auf mehreren Prozessoren zu ermöglichen.

\bigskip
Die im Rahmen der vorliegenden Arbeit erreichte Effizienzsteigerung ermöglicht zudem die Berechnung von Ringströmen zur Untersuchung magnetischer Eigenschaften in großen toroidalen Kohlenstoff-Nanoröhren (englisch \acfp{tcnt}) mit bis zu ca. 1000 Kohlenstoffatomen. Das sind nach meinem Kenntnisstand die bislang größten Systeme, für welche Ringströme berechnet wurden. 

Es zeigte sich, dass ein möglichst kleines HOMO-LUMO-Gap eine notwendige -- aber nicht hinreichende -- Bedingung für das Vorhandensein eines Gesamtringstromes  ist. Dadurch werden insbesondere metallische \acp{tcnt} zu vielversprechenden Kandidaten. In der Tat wurden für letztere große (vorwiegend diatropische) Ringströme berechnet, aber auch für die von Dunlap vorgeschlagenen $D_{\text{6h}}$-symmetrischen \acp{tcnt}. Grundsätzlich hängen Ringströme stark von den strukturellen und elektronischen Eigenschaften der Systeme ab, beispielsweise von der Orientierung und Lage der fünf- und siebengliedrigen Ringe. Bei den Untersuchungen konnte außerdem gezeigt werden, dass der Ringstrom mit zunehmendem Torusdurchmesser an Stärke gewinnt. 
\vfill
\newpage
Ringströme wurden ebenfalls für das Anion [Hg$_8$Te$_8$(Te$_2$)$_4$]$^{8-}$ berechnet, welches in der Arbeitsgruppe Dehnen (Universität Marburg) synthetisiert wurde und dessen Struktur an das Porphyrin erinnert. Es zeigte sich, dass bei diesem Molekül, anders als bei Porphyrin, kein globaler Ringstrom auftritt, sondern nur schwache lokale Ringströme in den fünfgliedrigen Ringen. Dies bedeutet, dass die Elektronen nicht über das gesamte Molekül delokalisiert sind und damit keine globale Aromatizität vorliegt. 

Zwei weitere Anwendungen des Programms auf anorganische Systeme beschäftigten sich mit der Erklärung von $^{31}$P-chemischen Verschiebungen sowie der Zuordnung von $^{119}$Sn-chemischen Verschiebungen in den endohedralen Clusteranionen [Co@Sn$_6$Sb$_6$]$^{3-}$ und [Co$_2$@Sn$_5$Sb$_7$]$^{3-}$.

\bigskip
Diese Beispiele zeigen, dass die in der vorliegenden Arbeit implementierten Methoden erfolgreich auf chemische Fragestellungen angewendet werden können. Näherungsverfahren sowie Programmoptimierungen und -parallelisierung erlauben die Berechnung von \ac{nmr}-Abschirmungstensoren für sehr große System mit vergleichsweise geringem Zeitaufwand. Die Erweiterungen der Funktionalität ermöglichen die Berechnung chemischer Verschiebungen von Atomen in Molekülen, welche schwere Elemente beinhalten oder hoch geladen sind. Umgebungseffekte können auf unterschiedliche Weise berücksichtigt werden und simulierte \ac{vcd}-Spektren ermöglichen die Bestimmung absoluter Konfigurationen in chiralen Molekülen.
\vfill
\newpage
\thispagestyle{empty}
\cleardoublepage