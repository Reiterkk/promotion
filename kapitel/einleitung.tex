Die Kernspinresonanzspektroskopie (englisch \ac{nmr} \textit{spectroscopy}) ist eine der wichtigsten analytischen Methoden bei der Strukturaufklärung von Molekülen. In einigen Fällen ist es dabei von Nutzen, experimentell gemessene Spektren mit berechneten chemischen Abschirmungskonstanten zu vergleichen, um die Signalzuordnung zu erleichtern oder überhaupt erst zu ermöglichen. Dies ist insbesondere dann der Fall, wenn unerwartete Signale im Spektrum auftreten oder die Spektren nicht ohne weitere Hilfsmittel interpretiert werden können. Kommen mehrere Verbindungen in Frage, können die chemischen Abschirmungskonstanten für diesen Molekültestsatz berechnet und die am besten zum Experiment passende Struktur ermittelt werden. Im Rahmen der Quantenchemie ist dafür eine sowohl akkurate als auch effiziente Methode notwendig. Die Grundlage dafür legte Ditchfield\supercite{ditchfield1974self} Mitte der 1970er-Jahre mit seiner eichursprungsinvarianten Implementierung. Wolinski, Hinton und Pulay\supercite{wolinski1990efficient} implementierten 1990 eine deutlich effizientere Methode auf Hartree-Fock-Niveau, welche die beiden oben genannten Punkte zum ersten Mal erfüllen konnte. Dies wurde durch eine effizient abgeschätzte \glqq \textit{on the fly}\grqq{}-Berechnung der Vierzentren-Zweielektronen-Integrale sowie durch das Vermeiden der Transformation dieser Integrale erreicht. Kurz darauf folgte die Entwicklung von Methoden zur Berechnung von chemischen Abschirmungskonstanten mit der \ac{mp2}\supercite{gauss1992calculation} und mit \ac{cc}\supercite{gauss1995gauge} von Gauss und Stanton. Für diese Methoden wurde eine hohe Genauigkeit berechneter chemischer Verschiebungen von $^1$H- und $^{13}$C-Kernen aus einem Testsatz mit kleinen organischen Molekülen demonstriert. Allerdings liegt der Rechenaufwand und Speicherbedarf deutlich über dem der Hartree-Fock-Rechnungen, sodass diese Methoden nicht auf größere Moleküle angewendet werden können. Eine deutlich effizientere Berechnung auf dem Niveau der \ac{dft} wurde etwa zur selben Zeit von Lee, Handy und Colwell\supercite{lee1995density} vorgestellt. 

\bigskip
Die im Rahmen der vorliegenden Arbeit durchgeführten Implementierungsarbeiten haben zwei wesentliche Ziele. Zum einen soll die Funktionalität des Moduls für die Berechnung chemischer Abschirmungskonstanten \texttt{mpshift}\supercite{haser1992direct,kollwitz1996direct} des Programmpakets \textsc{Turbomole}\supercite{ahlrichs1989electronic,TURBOMOLE,furche2014turbomole} erweitert werden. Zum anderen soll die Effizienz dieses Moduls gesteigert werden, um die Rechnungen für große Systeme zu ermöglichen. Die erweiterte Funktionalität soll dabei die Berechnung von chemischen Abschirmungskonstanten für anionische Verbindungen sowie für Verbindungen, welche schwere Elemente (Kernladung > 36) beinhalten, ermöglichen und Umgebungseffekte bei der Berechnung einbeziehen. Dafür ist es notwendig, das \ac{cosmo}\supercite{klamt1993cosmo} und die sogenannten \acp{ecp} für die Berechnung der chemischen Abschirmungskonstanten zu implementieren. Neben der \ac{nmr}-Spektroskopie gewann auch die \mbox{\aclu*{vcd-}(\acs{vcd}-)}\acused{vcd}Spektroskopie in der näheren Vergangenheit deutlich an Popularität. Mit ihrer Hilfe und insbesondere durch den notwendigen Vergleich von gemessenen und berechneten Spektren lassen sich absolute Konfigurationen in Molekülen bestimmen. Bei ihrer Berechnung wird ebenfalls die Antwort der Wellenfunktion des Moleküls auf ein externes Magnetfeld (\textit{magnetic response}) benötigt. Um \ac{vcd}-Spektren berechnen zu können, ist es daher notwendig, die entsprechenden Gleichungen in das Programmpaket zu implementieren. 

Weiterhin soll die Effizienz des Moduls \texttt{mpshift} deutlich verbessert werden, um die Berechnung chemischer Abschirmungskonstanten oder \ac{vcd}-Spektren in großen Molekülen zu realisieren. Dafür ist es notwendig, die \aclu*{ri-}\mbox{(\acs{ri}-)}\acused{ri}Methode\supercite{vahtras1993integral} auf die nach den Komponenten des Magnetfeldes abgeleiteten Vierzentren-Zweielektronen-Integrale zu übertragen. Eine zusätzliche Steigerung der Effizienz soll durch die Adaption des \aclu*{marij-}\mbox{(\acs{marij}-)}\acused{marij}Verfahrens\supercite{sierka2003fast} erreicht werden. Die \ac{ri}-Methode und die zusätzliche Beschleunigung durch die Multipolnäherung können bereits bei der Berechnung des Coulombbeitrages für die Wellenfunktion des elektronischen Grundzustandes verwendet werden und versprechen auch bei der Berechnung chemischer Abschirmungskonstanten einen deutlichen Effizienzgewinn. Durch das Anwenden dieser Näherungsverfahren wird der Hartree-Fock-Austausch unabhängig vom Coulombbeitrag berechnet. Da die Austauschwechselwirkung schneller mit dem Abstand abfällt als die Coulombwechselwirkung, kann zusätzlich eine effizientere Integralabschätzung für die Berechnung des Austauschbeitrages implementiert werden. Weitere Beschleunigungen der Berechnungen sollen durch die Implementierung einer moderaten OpenMP-Parallelisierung sowie durch gezielte Optimierung des Programmcodes gewährleistet werden. 

\bigskip
Wird ein Molekül einem äußeren Magnetfeld ausgesetzt, kommt es zur Induktion von diatropischen und paratropischen Ringströmen.\supercite{taubert2011calculation} In aromatischen Verbindungen dominiert der im Bezug auf die Magnetfeldrichtung im Uhrzeigersinn fließende diatropische Beitrag und ein diatropischer Gesamtringstrom resultiert. Im Gegensatz dazu wird für antiaromatische Verbindungen ein gegen den Uhrzeigersinn fließender paratropischer Gesamtringstrom erhalten. Sind die diatropischen und paratropischen Beiträge von gleicher Größe, heben sie sich gegenseitig auf und die Verbindung ist nichtaromatisch. Die Stärke des Gesamtringstroms stellt ein Maß für die Delokalisierung der Elektronen bzw. die Aromatizität der Verbindung dar.\supercite{elvidge1961181,pople1966induced} Typisch aromatische Verbindungen wie Benzol oder Porphyrin besitzen beispielsweise diatropische Gesamtringströme von etwa \unit[12]{nA/T} bzw. \unit[27]{nA/T}.\supercite{fliegl2012aromatic} Neben der Berechnung von \ac{nmr}- und zukünftig \ac{vcd}-Spektren kann das Modul \texttt{mpshift} auch die nach den Komponenten des Magnetfeldes abgeleitete Elektronendichte zur Verfügung stellen. Diese wird von dem eigenständigen Programm GIMIC (\acl{gimic})\supercite{juselius2004calculation,taubert2011calculation,fliegl2011gauge,sundholm2016calculations} zur Berechnung der magnetisch induzierten Stromdichte und der Ringströme benötigt.

\bigskip
Schließlich sollen die im Rahmen der vorliegenden Arbeit implementierten Methoden auf chemische Fragestellungen angewendet werden. Zusätzlich zur schnellen Berechnung von chemischen Abschirmungskonstanten in großen Molekülen ermöglicht die gesteigerte Effizienz eine Untersuchung der magnetischen Eigenschaften von großen toroidalen Kohlenstoff-Nanoröhren mit 1000 und mehr Atomen auf \ac{dft}-Niveau. Aufgrund ihrer einzigartigen Geometrie stellen diese Verbindungen sowohl auf experimenteller aber auch auf theoretischer Ebene ein aktuelles Forschungsgebiet dar. Im Hinblick auf mögliche zukünftige Anwendungsgebiete ist ein fundamentales Verständnis ihrer Eigenschaften wichtig. Durch die Berechnung von Ringströmen in diesen Systemen lassen sich Aussagen über die Delokalisierung der Elektronen darin treffen und Einflüsse struktureller Parameter beschreiben. 

Die Implementierung des \ac{cosmo} ermöglicht die Untersuchung der magnetischen Eigenschaften und die Berechnung von \ac{nmr}-Spektren in anionischen anorganischen Verbindungen. Mit seiner Hilfe soll der aromatische Charakter des in der Arbeitsgruppe von Stefanie Dehnen synthetisierten Anions [Hg$_8$Te$_8$(Te$_2$)$_4$]$^{8-}$ untersucht werden, welches eine große strukturelle Ähnlichkeit mit dem organischen Porphyrin besitzt. Weiter sollen experimentell gemessene \ac{nmr}-Spektren von endohedralen Zinn-Antimon-Clusteranionen mithilfe berechneter chemischer Verschiebungen gedeutet werden und eine Zuordnung der Signale auf die atomaren Positionen erfolgen. 

\bigskip
Die vorliegende Arbeit ist wie folgt aufgebaut: Kapitel \ref{theorie} beinhaltet die allgemeinen und für diese Arbeit notwendigen theoretischen Grundlagen. Neben einer kurzen Zusammenfassung des Hartree-Fock-Verfahrens und der \acl{dft} ist dort eine ausführliche Herleitung der notwendigen Gleichungen für die Berechnung chemischer Abschirmungskonstanten auf diesen beiden Niveaus gegeben. Zusätzlich wird am Ende des Kapitels kurz auf die Berechnung von Ringströmen und der Stromdichte mit dem Programm GIMIC eingegangen. Eine schematische Darstellung der Programmstruktur des Moduls \texttt{mpshift}, wie es vor den Änderungen, die im Rahmen dieser Arbeit durchgeführt werden, vorlag, ist in Kapitel \ref{programmstruktur} gezeigt. Dort wird die Funktion der wichtigsten Routinen zur Berechnung chemischer Abschirmungskonstanten erläutert. Die darauffolgenden Kapitel \ref{funktionalität} und \ref{effizienz} beschreiben die Implementierungen in das Programmpaket \textsc{Turbomole}, welche im Rahmen dieser Arbeit durchgeführt werden. In den darin enthaltenen Unterkapiteln wird zunächst eine kurze Einführung in die jeweilige Theorie gegeben, bevor die eigentliche Implementierung im Detail erläutert wird. Sofern es von Relevanz ist, wird dort ebenfalls eine schematische Darstellung der Programmstruktur mit den neu implementierten oder modifizierten Routinen gezeigt. Ausgewählte Testrechnungen veranschaulichen die neu implementierten Funktionalitäten. In Kapitel \ref{genauigkeit} wird auf Genauigkeitseinbußen und die Effizienzsteigerung der implementierten Näherungsverfahren zur Verkürzung der Rechenzeit eingegangen. Zur Untersuchung magnetischer Eigenschaften werden in Kapitel \ref{anwendungen} die in der vorliegenden Arbeit vorgenommenen Implementierungsarbeiten angewendet. Mit ihrer Hilfe sollen chemische Fragestellungen beantwortet und ein besseres Verständnis magnetischer Eigenschaften erhalten werden. Eine abschließende Zusammenfassung dieser Arbeit erfolgt in Kapitel \ref{zusammenfassung}, bevor in Kapitel \ref{ausblick} ein kurzer Ausblick auf eine mögliche Implementierung der semi-numerischen Berechnung des Hartree-Fock-Austauschs und auf das \aclu*{rik-}\mbox{(\acs{rik}-)}\acused{rik}Verfahren\supercite{weigend2002fully} für chemische Abschirmungskonstanten gegeben wird. Die hierfür notwendigen Gleichungen sind dafür ausgearbeitet worden und eine Möglichkeit zur Implementierung ist dargelegt.