Die \ac{nmr} Spektroskopie ist eine der wichtigsten analytischen Methoden bei der Strukturaufklärung von Molekülen. In einigen Fällen kann es von Nutzen sein, gemessene Spektren mit berechneten chemischen Abschirmungskonstanten zu vergleichen. Dies ist insbesondere dann der Fall, wenn unerwartete Signale im Spektrum auftreten oder die Spektren nicht ohne weitere Hilfsmittel gedeutet werden können. Kommen mehrere Verbindungen in Frage, können die chemischen Abschirmungskonstanten für einen Testsatz an Molekülen berechnet und die am besten zum Experiment passende Struktur ermittelt werden. Im Rahmen der Quantenchemie wird dafür sowohl eine akkurate aber auch effiziente Methode notwendig. Wolinski, Hinton und Pulay\supercite{wolinski1990efficient} implementierten 1990 eine Methode auf Hatree-Fock-Niveau, welche die beiden Punkte zum ersten mal genau genug erfüllen konnte. 

Das Ziel der vorliegenden Arbeit kann in zwei wesentliche Punkte aufgeteilt werden. Zum einen sollte die Funktionalität des Moduls \texttt{mpshift}\supercite{haser1992direct,kollwitz1996direct} des Programmpaketes \textsc{TURBOMOLE}\supercite{ahlrichs1989electronic,TURBOMOLE,furche2014turbomole} erweitert und zum anderen die Effizienz bei der Berechnung gesteigert werden. Die erweiterte Funktionalität sollte dabei die Berechnung von chemischen Abschirmungskonstanten für anionische Verbindungen, sowie für Verbindungen welche schwere Elemente (Kernladung > 36) beinhalten, überhaupt erst ermöglichen, Umgebungseffekte bei der Berechnung mit einbeziehen und die Möglichkeit zur Berechnung von \ac{vcd} Spektren beinhalten. Um dies zu ermöglichen war es notwendig, das \ac{cosmo},\supercite{klamt1993cosmo} die \acp{ecp} für die Berechnung der chemischen Abschirmkonstanten und die Gleichungen für die Berechnung von \ac{vcd}-Spektren zu implementieren. 

Zur Berechnung der chemischen Abschirmkonstanten in großen Molekülen sollte im Weiteren die Effizienz des Moduls \texttt{mpshift} deutlich verbessert werden. Dafür war es notwendig, die \ac{ri}-Methode auf die nach dem Magnetfeld abgeleiteten Vierzentrenzweielektronen-Integrale zu übertragen. Eine zusätzliche Steigerung der Effizienz sollte durch die Adaption des \ac{marij}-Verfahrens erreicht werden. Die \ac{ri}-Methode und die zusätzliche Multipolnäherung können bereits bei der Berechnung des Coulombbeitrages für die Wellenfunktion des elektronischen Grundzustandes angewendet werden und versprechen auch bei der Berechnung chemischer Abschirmungskonstanten einen deutlichen Effizienzgewinn. Weitere Beschleunigungen der Berechnungen sollten durch die Implementierung einer moderaten OpenMP-Parallelisierung sowie durch gezielte Optimierung des Programmcodes gewährleistet werden. 

\bigskip
Die vorliegende Arbeit ist wie folgt aufgebaut. Kapitel \ref{theorie} beinhaltet die notwendigen theoretischen Grundlagen für diese Arbeit. In den Kapiteln \ref{effizienz} und \ref{funktionalität} wird die Implementierung in das Programmpaket \textsc{TURBOMOLE} beschrieben. In Kapitel \ref{anwendungen} werden exemplarische Anwendungsbeispiele vorgestellt, welche durch diese Arbeit ermöglicht wurden. Eine Zusammenfassung dieser Arbeit erfolgt in Kapitel \ref{zusammenfassung}.