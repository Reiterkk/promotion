Die \ac{nmr} Spektroskopie ist eine der wichtigsten analytischen Methoden bei der Strukturaufklärung von Molekülen. In einigen Fällen kann es dabei von Nutzen sein, gemessene Spektren mit berechneten chemischen Abschirmungskonstanten zu vergleichen um die Signalzuordnung zu erleichtern. Dies ist insbesondere dann der Fall, wenn unerwartete Signale im Spektrum auftreten oder die Spektren nicht ohne weitere Hilfsmittel interpretiert werden können. Kommen mehrere Verbindungen in Frage, können die chemischen Abschirmungskonstanten für einen Testsatz an Molekülen berechnet und die am besten zum Experiment passende Struktur ermittelt werden. Im Rahmen der Quantenchemie wird dafür sowohl eine akkurate aber auch effiziente Methode notwendig. Die Grundlage dafür legte Ditchfield\supercite{ditchfield1974self} Mitte der 1970er Jahre mit seiner Eichursprung invarianten Implementierung. Wolinski, Hinton und Pulay\supercite{wolinski1990efficient} implementierten 1990 eine deutlich effizientere Methode auf Hatree-Fock-Niveau, welche die oben genannten beiden Punkte zum ersten mal genau genug erfüllen konnte. Dies wurde durch eine effiziente Berechnung der zwei-Elektronen-Integrale und der Vermeidung der Speicherung dieser, einer effizienten Integralabschätzung sowie dem Vermeiden der Transformation der vier-Zentren-zwei-Elektonen-Integrale erreicht. Kurz darauf folgten die Berechnungen auf dem Level der \ac{mp2}\supercite{gauss1992calculation} sowie auf \ac{cc}\supercite{gauss1995gauge} Niveau von Gauss und Stanton. Für diese Methoden wurde eine hohe Genauigkeit der $^1$H und $^{13}$C Verschiebungen für einen Testsatz mit kleinen organischen Molekülen demonstriert. Allerdings liegt der Rechenaufwand und Speicherbedarf deutlich über dem der Hartree-Fock-Rechnungen, sodass diese Methoden nicht auf größere Moleküle angewendet werden können. Eine deutlich effizientere Berechnung auf \ac{dft} Niveau wurde etwa zur selben Zeit von Lee, Handy und Colwell\supercite{lee1995density} vorgestellt. 

\bigskip
Das Ziel der vorliegenden Arbeit kann in zwei wesentliche Punkte aufgeteilt werden. Zum einen sollte die Funktionalität des Moduls für die Berechnung chemischer Abschirmungskonstanten \texttt{mpshift}\supercite{haser1992direct,kollwitz1996direct} des Programmpaketes \textsc{TURBOMOLE}\supercite{ahlrichs1989electronic,TURBOMOLE,furche2014turbomole} erweitert werden. Zum anderen sollte die Effizienz bei der Berechnung gesteigert werden, um die Berechnung größerer Systeme in einer akzeptablen Zeit zu ermöglichen. Die erweiterte Funktionalität sollte dabei die Berechnung von chemischen Abschirmungskonstanten für anionische Verbindungen, sowie für Verbindungen, welche schwere Elemente (Kernladung > 36) beinhalten, überhaupt erst ermöglichen und Umgebungseffekte bei der Berechnung mit einbeziehen. Um dies zu ermöglichen war es notwendig, das \ac{cosmo},\supercite{klamt1993cosmo} die \acp{ecp} für die Berechnung der chemischen Abschirmkonstanten zu implementieren. Neben der \ac{nmr} Spektroskopie hat auch die \ac{vcd} Spektroskopie in der näheren Vergangenheit an deutlich an Popularität gewonnen. Mit ihrer Hilfe und insbesondere dem notwendigen Vergleich von gemessenen und berechneten Spektren lassen sich die absoluten Konfigurationen in Molekülen bestimmen. Bei Ihrer Berechnung wird ebenfalls die Response der Wellenfunktion des Moleküls auf ein externes Magnetfeld benötigt. Um die \ac{vcd} Spektren berechnen zu können, sollten daher auch die notwendigen Gleichungen in das Programmpaket implementiert werden. 

Zur Berechnung der chemischen Abschirmkonstanten oder \ac{vcd} Spektren in großen Molekülen sollte im Weiteren die Effizienz des Moduls \texttt{mpshift} deutlich verbessert werden. Dafür war es notwendig, die \ac{ri}-Methode auf die nach dem Magnetfeld abgeleiteten vier-Zentren-zwei-Elektronen-Integrale zu übertragen. Eine zusätzliche Steigerung der Effizienz sollte durch die Adaption des \ac{marij}-Verfahrens\supercite{sierka2003fast} erreicht werden. Die \ac{ri}-Methode und die zusätzliche Beschleunigung durch die Multipolnäherung können bereits bei der Berechnung des Coulombbeitrages für die Wellenfunktion des elektronischen Grundzustandes verwendet werden und versprechen auch bei der Berechnung chemischer Abschirmungskonstanten einen deutlichen Effizienzgewinn. Weitere Beschleunigungen der Berechnungen sollten durch die Implementierung einer moderaten OpenMP-Parallelisierung sowie durch gezielte Optimierung des Programmcodes gewährleistet werden. 

\bigskip
Die vorliegende Arbeit ist wie folgt aufgebaut. Kapitel \ref{theorie} beinhaltet die allgemeinen notwendigen theoretischen Grundlagen für diese Arbeit. Neben einer kurzen Zusammenfassung des Hartree-Fock-Verfahrens und der \ac{dft} ist dort eine ausführliche Herleitung der notwendigen Gleichungen für die Berechnung der chemischen Abschirmungskonstanten auf diesen beiden Niveaus gegeben. Zusätzlich wird am Ende des Kapitels kurz auf die Berechnung von Ringströmen mit dem Programm GIMIC\supercite{juselius2004calculation,taubert2011calculation,fliegl2011gauge,sundholm2016calculations} eingegangen. Eine schematische Programmstruktur des Moduls \texttt{mpshift}, wie es vor den Änderungen die in dieser Arbeit gemacht wurden ist in Kapitel \ref{programmstruktur} dargestellt. Dort wird die Funktion der wichtigsten Routinen bei der Berechnung erläutert. Die darauffolgenden Kapitel \ref{effizienz} und \ref{funktionalität} beschreiben die Implementierungen in das Programmpaket \textsc{TURBOMOLE} die im Rahmen dieser Arbeit durchgeführt wurden. In den darin enthaltenen Unterkapiteln ist zu nächst eine kurze Einführung in die jeweilige Theorie beschrieben bevor die eigentliche Implementierung im Detail erläutert wird. Sofern es von Relevanz ist, wird dort ebenfalls eine schematische Darstellung der Programmstruktur mit den neu implementierten oder modifizierten Routinen gezeigt. Ausgewählte Testrechnungen veranschaulichen die neu implementierten Funktionalitäten. Im Kapitel \ref{genauigkeit} wird auf die Genauigkeit und die Effizienz der implementierten Näherungsverfahren zur Verkürzung der Rechenzeit eingegangen. Schließlich sind in Kapitel \ref{anwendungen} einige reale Anwendungsbeispiele gezeigt auf die die  neuen Entwicklungen, welche durch diese Arbeit ermöglicht wurden, angewendet werden. Eine abschließende Zusammenfassung dieser Arbeit erfolgt in Kapitel \ref{zusammenfassung}.