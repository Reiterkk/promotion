Das Ziel der vorliegenden Arbeit war es die Funktionalität und die Effizienz des Moduls \texttt{mpshift} des Programmpakets \textsc{TURBOMOLE}\supercite{TURBOMOLE,grimme2010consistent,born1927quantentheorie} zu verbessern. Die erweiterte Funktionalität sollte dabei die Berechnung von chemischen Abschirmungskonstanten für anionische Verbindungen, sowie Verbindungen welche schwere Elemente (Kernladung > 36) beinhalten, überhaupt erst ermöglichen, Umgebungseffekte bei der Berechnung mit einbeziehen und die Möglichkeit zur Berechnung von \ac{vcd} Spektren beinhalten. Um dies zu ermöglichen war es notwendig, das \ac{cosmo}, die \acp{ecp} für die Berechnung der chemischen Abschirmkonstante und die Gleichungen für die Berechnung von \ac{vcd}-Spektren zu implementieren. 

Zur Berechnung der chemischen Abschirmkonstanten in großen Molekülen sollte im Weiteren die Effizienz des Moduls \texttt{mpshift} deutlich verbessert werden. Dafür sollte die \ac{ri}-Methode auf die nach dem Magnetfeld abgeleiteten Vier-Zentren-Zwei-Elektronen-Integrale übertragen werden. Eine zusätzliche Steigerung der Effizienz sollte durch die Adaption des \ac{marij}-Verfahrens erreicht werden. Weitere Beschleunigungen der Berechnungen sollten durch die Implementierung einer moderaten OpenMP-Parallelisierung sowie durch gezielte Optimierung des Programmcodes gewährleistet werden. 

\bigskip
Die vorliegende Arbeit ist wie folgt aufgebaut. Kapitel \ref{chap2} beinhaltet die notwendigen theoretischen Grundlagen für diese Arbeit. In Kapitel \ref{chap3} wird die Implementierung in das Programmpaket \textsc{TURBOMOLE} beschrieben. Kapitel \ref{chap4} dient zur Veranschaulichung der verbesserten Effizienz als auch zur Überprüfung der Genauigkeit der Implementierten Näherungsverfahren. In Kapitel \ref{chap5} werden exemplarische Anwendungsbeispiele vorgestellt, welche durch diese Arbeit ermöglicht wurden. Eine Zusammenfassung dieser Arbeit erfolgt in Kapitel \ref{chap6}.