Für die quantenchemische Berechnung von Molekülen im Magnetfeld ist die Ableitung der Energie nach dem Magnetfeld $\vec{B}$, $\frac{\partial E}{\partial \vec{B}}$, von zentraler Bedeutung. Da die Energie in der Quantenchemie ein Funktional der Wellenfunktion oder der Elektronendichte ist, wird hierfür die Berechnung der Ableitung dieser Größen nach einem äußeren Magnetfeld benötigt. Daraus lassen sich direkt Ringströme berechnen, welche als Maß für die Aromatizität verwendet werden können. Die zusätzliche Ableitung der Energie nach den magnetischen Momenten der Kerne $\vec{\mu}$, also die gemischte zweite Ableitung $\frac{\partial^2 E}{\partial \vec{B}\partial\vec{\mu}}$, definiert die Abschirmungstensoren $\sigma$, aus der gemischten dritten Ableitung nach dem Magnetfeld und den Kernkoordinaten $R$, $\frac{\partial^2 E}{\partial \vec{B}\partial^2 R}$, lassen sich sogenannte \mbox{\aclu*{vcd-}(\acs{vcd}-)}\acused{vcd}Spektren berechnen. 

Insbesondere die Berechnung der Abschirmungstensoren hat in der Quantenchemie eine lange Tradition, da die experimentelle Kernspinresonanzspektroskopie (englisch \ac{nmr} \textit{spectroscopy}), mit den daraus erhaltenen chemischen Verschiebungen, eine der wichtigsten analytischen Methoden bei der Strukturaufklärung von Molekülen ist. Oft ist es dabei von Nutzen, experimentell gemessene Spektren mit berechneten Abschirmungskonstanten zu vergleichen, um die Signalzuordnung zu erleichtern oder überhaupt erst zu ermöglichen. Dies ist insbesondere dann der Fall, wenn unerwartete Signale im Spektrum auftreten oder die Spektren nicht ohne weitere Hilfsmittel interpretiert werden können. Kommen mehrere Verbindungen in Frage, können die Abschirmungskonstanten für diese berechnet und die am besten zum Experiment passende Struktur ermittelt werden. Im Rahmen der Quantenchemie ist dafür eine sowohl akkurate als auch effiziente Methode notwendig. Die theoretische Grundlage dafür legte Ditchfield\supercite{ditchfield1974self} Mitte der 1970er-Jahre mit seiner eichursprungsinvarianten Implementierung. Wolinski, Hinton und Pulay\supercite{wolinski1990efficient} implementierten 1990 eine deutlich effizientere Methode auf Hartree-Fock-Niveau, welche die beiden oben genannten Punkte zum ersten Mal erfüllen konnte. Dies wurde durch eine effizient abgeschätzte \glqq \textit{on the fly}\grqq{}-Berechnung der Vierzentren-Zweielektronen-Integrale sowie durch das Vermeiden der Transformation dieser Integrale erreicht. Kurz darauf folgte die Entwicklung von Methoden zur Berechnung von Abschirmungskonstanten mit der \ac{mp2}\supercite{gauss1992calculation} und mit \ac{cc}\supercite{gauss1995gauge} von Gauss und Stanton. Für diese Methoden wurde eine hohe Genauigkeit berechneter chemischer Verschiebungen von $^1$H- und $^{13}$C-Kernen aus einem Testsatz mit kleinen organischen Molekülen demonstriert. Allerdings liegt der Rechenaufwand und Speicherbedarf deutlich über dem der Hartree-Fock-Rechnungen, sodass diese Methoden nicht auf größere Moleküle angewendet werden können. Eine deutlich effizientere Berechnung auf dem Niveau der \ac{dft} wurde etwa zur selben Zeit von Lee, Handy und Colwell\supercite{lee1995density} vorgestellt. 

Neben der \ac{nmr}-Spektroskopie gewann auch die \ac{vcd}-Spektroskopie in der jüngeren Vergangenheit deutlich an Popularität. Mit ihrer Hilfe und insbesondere durch den notwendigen Vergleich von gemessenen und berechneten Spektren lassen sich absolute Konfigurationen in Molekülen bestimmen. Bei ihrer Berechnung wird die Antwort der Wellenfunktion des Moleküls auf ein externes Magnetfeld (\textit{magnetic response}) benötigt. Aus dieser lassen sich ebenfalls, mit verhältnismäßig geringem Aufwand, sogenannte Ringströme berechnen, die in einem Molekül induziert werden, wenn es einem äußeren Magnetfeld ausgesetzt wird.\supercite{taubert2011calculation} In aromatischen Verbindungen dominiert der, im Bezug auf die Magnetfeldrichtung im Uhrzeigersinn fließende, diatropische Beitrag und ein diatropischer Gesamtringstrom resultiert. Im Gegensatz dazu wird für antiaromatische Verbindungen ein gegen den Uhrzeigersinn fließender paratropischer Gesamtringstrom erhalten. Sind die diatropischen und paratropischen Beiträge von gleicher Größe, heben sie sich gegenseitig auf und die Verbindung ist nichtaromatisch. Die Stärke des Gesamtringstroms stellt ein Maß für die Delokalisierung der Elektronen bzw. die Aromatizität der Verbindung dar.\supercite{elvidge1961181,pople1966induced} Typisch aromatische Verbindungen wie Benzol oder Porphyrin besitzen beispielsweise diatropische Gesamtringströme von etwa \unit[12]{nA/T} bzw. \unit[27]{nA/T}.\supercite{fliegl2012aromatic}
Die Berechnung dieser Ringströme kann mit dem eigenständigen Programm GIMIC (\acl{gimic})\supercite{juselius2004calculation,taubert2011calculation,fliegl2011gauge,sundholm2016calculations} erfolgen, welches hierfür die zuvor mit \texttt{mpshift} berechnete abgeleitete Elektronendichte benutzt. In allen Fällen ist daher eine effiziente Berechnung der \textit{magnetic response} erforderlich.


\bigskip
Üblicherweise werden die \ac{nmr}- und \ac{vcd}-Spektren der zu analysierenden Verbindungen nicht in der Gasphase, sondern in Lösung gemessen. Daher ist es notwendig, Lösungsmitteleffekte in die Berechnungen einzubeziehen. Zusätzlich können in Verbindungen, welche schwere Elemente (Kernladung > 36) beinhalten, relativistische Effekte auftreten, die bei der Berechnung ebenfalls berücksichtigt werden müssen. Aus diesem Grund haben die im Rahmen der vorliegenden Arbeit durchgeführten Implementierungsarbeiten zwei wesentliche Ziele. Das erste ist die Erweiterung der Funktionalität des Moduls \texttt{mpshift}\supercite{haser1992direct,kollwitz1996direct} aus dem  \textsc{Turbomole}-Programmpaket\supercite{ahlrichs1989electronic,TURBOMOLE,furche2014turbomole} für die Berechnung der Abschirmungskonstanten. Zur Berücksichtigung von Umgebungseffekten und um die Berechnungen für anionische Verbindungen zu ermöglichen, ist es notwendig, das \ac{cosmo}\supercite{klamt1993cosmo} für die Berechnung der Abschirmungskonstanten zu implementieren. Die Implementierung von sogenannten \acp{ecp} ermöglicht zudem eine Beschreibung der relativistischen Einflüsse auf benachbarte Atome von schweren Elementen.

Das zweite Ziel ist eine deutliche Verbesserung der Effizienz des Moduls \texttt{mpshift}, um die Berechnungen der \textit{magnetic response} für große Moleküle zu realisieren. Die bei der Berechnung des Coulombbeitrages für den elektronischen Grundzustand verwendete \aclu*{ri-}\mbox{(\acs{ri}-)}\acused{ri}Methode\supercite{vahtras1993integral} und die zusätzliche Beschleunigung durch eine Multipolnäherung, das \aclu*{marij-}\mbox{(\acs{marij}-)}\acused{marij}Verfahren\supercite{sierka2003fast}, versprechen auch bei der Berechnung der Abschirmungskonstanten einen erheblichen Effizienzgewinn. Hierfür ist es notwendig, die \ac{marij}-Methode auf die nach den Komponenten des Magnetfeldes abgeleiteten Vierzentren-Zweielektronen-Integrale zu übertragen. Durch das Anwenden dieser Näherungsverfahren wird der Hartree-Fock-Austausch unabhängig vom Coulombbeitrag berechnet. Da die Austauschwechselwirkung schneller mit dem Abstand abfällt als die Coulombwechselwirkung, kann zusätzlich eine effizientere Integralabschätzung für die Berechnung des Austauschbeitrages implementiert werden. Weitere Beschleunigungen der Berechnungen sollen durch die Implementierung einer moderaten OpenMP-Parallelisierung sowie durch gezielte Optimierung des Programmcodes gewährleistet werden. 

\bigskip
Die vorliegende Arbeit ist wie folgt aufgebaut: Kapitel \ref{theorie} fasst die allgemeinen und für diese Arbeit notwendigen theoretischen Grundlagen zusammen. Eine schematische Darstellung der Programmstruktur des Moduls \texttt{mpshift}, wie es vor den Änderungen, die im Rahmen dieser Arbeit durchgeführt werden, vorlag, ist in Kapitel \ref{programmstruktur} gezeigt. Darin wird ebenfalls die Funktion der wichtigsten Routinen zur Berechnung der Abschirmungskonstanten erläutert. Die darauffolgenden Kapitel \ref{funktionalität} und \ref{effizienz} beschreiben die Implementierungen in das Programmpaket \textsc{Turbomole}, welche im Rahmen dieser Arbeit durchgeführt werden. In den darin enthaltenen Abschnitten wird zunächst eine kurze Einführung in die jeweilige Theorie gegeben, bevor die eigentliche Implementierung im Detail erläutert wird. Die in Kapitel \ref{genauigkeit} dokumentierten Testergebnisse dienen zur Validierung und zur Illustration der Implementierungsarbeiten. Kapitel \ref{anwendungen} beschäftigt sich mit einigen chemischen Fragestellungen, die durch die Weiterentwicklungen in Effizienz und Funktionalität bearbeitbar wurden. Beispielsweise ermöglicht die gesteigerte Effizienz eine Untersuchung der magnetischen Eigenschaften von großen toroidalen Kohlenstoff-Nanoröhren mit 1000 und mehr Atomen auf \ac{dft}-Niveau. Aufgrund ihrer einzigartigen Geometrie stellen diese Verbindungen sowohl auf experimenteller aber auch auf theoretischer Ebene ein aktuelles Forschungsgebiet dar. Im Hinblick auf mögliche zukünftige Anwendungsgebiete ist ein fundamentales Verständnis ihrer Eigenschaften wichtig. Durch die Berechnung von Ringströmen in diesen Systemen lassen sich Aussagen über die Delokalisierung der Elektronen darin treffen und Einflüsse struktureller Parameter beschreiben. 

Die Implementierung von \ac{cosmo} ermöglicht die Untersuchung der magnetischen Eigenschaften und die Berechnung von \ac{nmr}-Spektren in anionischen anorganischen Verbindungen. Mit seiner Hilfe soll der aromatische Charakter des in der Arbeitsgruppe Dehnen (Universität Marburg) synthetisierten Anions [Hg$_8$Te$_8$(Te$_2$)$_4$]$^{8-}$ untersucht werden, welches eine große strukturelle Ähnlichkeit mit Porphyrin besitzt. Weiter sollen experimentell gemessene \ac{nmr}-Spektren von endohedralen Zinn-Antimon-Clusteranionen mithilfe berechneter chemischer Verschiebungen gedeutet werden. Eine abschließende Zusammenfassung dieser Arbeit erfolgt in Kapitel \ref{zusammenfassung}, ein kurzer Ausblick in Kapitel \ref{ausblick}.

%\bigskip
%Die vorliegende Arbeit ist wie folgt aufgebaut: Kapitel \ref{theorie} beinhaltet die allgemeinen und für diese Arbeit notwendigen theoretischen Grundlagen. Neben einer kurzen Zusammenfassung des Hartree-Fock-Verfahrens und der \acl{dft} ist dort eine ausführliche Herleitung der notwendigen Gleichungen für die Berechnung chemischer Abschirmungskonstanten auf diesen beiden Niveaus gegeben. Zusätzlich wird am Ende des Kapitels kurz auf die Berechnung von Ringströmen und der Stromdichte mit dem Programm GIMIC eingegangen. Eine schematische Darstellung der Programmstruktur des Moduls \texttt{mpshift}, wie es vor den Änderungen, die im Rahmen dieser Arbeit durchgeführt werden, vorlag, ist in Kapitel \ref{programmstruktur} gezeigt. Dort wird die Funktion der wichtigsten Routinen zur Berechnung chemischer Abschirmungskonstanten erläutert. Die darauffolgenden Kapitel \ref{funktionalität} und \ref{effizienz} beschreiben die Implementierungen in das Programmpaket \textsc{Turbomole}, welche im Rahmen dieser Arbeit durchgeführt werden. In den darin enthaltenen Unterkapiteln wird zunächst eine kurze Einführung in die jeweilige Theorie gegeben, bevor die eigentliche Implementierung im Detail erläutert wird. Sofern es von Relevanz ist, wird dort ebenfalls eine schematische Darstellung der Programmstruktur mit den neu implementierten oder modifizierten Routinen gezeigt. Ausgewählte Testrechnungen veranschaulichen die neu implementierten Funktionalitäten. In Kapitel \ref{genauigkeit} wird auf Genauigkeitseinbußen und die Effizienzsteigerung der implementierten Näherungsverfahren zur Verkürzung der Rechenzeit eingegangen. Zur Untersuchung magnetischer Eigenschaften werden in Kapitel \ref{anwendungen} die in der vorliegenden Arbeit vorgenommenen Implementierungsarbeiten angewendet. Mit ihrer Hilfe sollen chemische Fragestellungen beantwortet und ein besseres Verständnis magnetischer Eigenschaften erhalten werden. Eine abschließende Zusammenfassung dieser Arbeit erfolgt in Kapitel \ref{zusammenfassung}, bevor in Kapitel \ref{ausblick} ein kurzer Ausblick auf eine mögliche Implementierung der semi-numerischen Berechnung des Hartree-Fock-Austauschs und auf das \aclu*{rik-}\mbox{(\acs{rik}-)}\acused{rik}Verfahren\supercite{weigend2002fully} für chemische Abschirmungskonstanten gegeben wird. Die hierfür notwendigen Gleichungen sind dafür ausgearbeitet worden und eine Möglichkeit zur Implementierung ist dargelegt.


%Dafür ist es notwendig, die \aclu*{ri-}\mbox{(\acs{ri}-)}\acused{ri}Methode\supercite{vahtras1993integral} auf die nach den Komponenten des Magnetfeldes abgeleiteten Vierzentren-Zweielektronen-Integrale zu übertragen. Eine zusätzliche Steigerung der Effizienz soll durch die Adaption des \aclu*{marij-}\mbox{(\acs{marij}-)}\acused{marij}Verfahrens\supercite{sierka2003fast} erreicht werden. Die \ac{ri}-Methode und die zusätzliche Beschleunigung durch die Multipolnäherung können bereits bei der Berechnung des Coulombbeitrages für die Wellenfunktion des elektronischen Grundzustandes verwendet werden und versprechen auch bei der Berechnung chemischer Abschirmungskonstanten einen deutlichen Effizienzgewinn.

%Wird ein Molekül einem äußeren Magnetfeld ausgesetzt, kommt es zur Induktion von diatropischen und paratropischen Ringströmen.\supercite{taubert2011calculation}

%Neben der Berechnung von \ac{nmr}- und zukünftig \ac{vcd}-Spektren kann das Modul \texttt{mpshift} auch die nach den Komponenten des Magnetfeldes abgeleitete Elektronendichte zur Verfügung stellen. Diese wird von dem eigenständigen Programm GIMIC (\acl{gimic})\supercite{juselius2004calculation,taubert2011calculation,fliegl2011gauge,sundholm2016calculations} zur Berechnung der magnetisch induzierten Stromdichte und der Ringströme benötigt.