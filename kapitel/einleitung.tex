Die \ac{nmr} Spektroskopie ist eine der wichtigsten analytischen Methoden bei der Strukturaufklärung von Molekülen. In einigen Fällen ist es dabei von Nutzen, experimentell gemessene Spektren mit berechneten chemischen Abschirmungskonstanten zu vergleichen um die Signalzuordnung zu erleichtern oder überhaupt erst zu ermöglichen. Dies ist insbesondere dann der Fall, wenn unerwartete Signale im Spektrum auftreten oder die Spektren nicht ohne weitere Hilfsmittel interpretiert werden können. Kommen mehrere Verbindungen in Frage, können die chemischen Abschirmungskonstanten für diesen Molekültestsatz berechnet und die am besten zum Experiment passende Struktur ermittelt werden. Im Rahmen der Quantenchemie ist dafür sowohl eine akkurate aber auch effiziente Methode notwendig. Die Grundlage dafür legte Ditchfield\supercite{ditchfield1974self} Mitte der 1970er Jahre mit seiner Eichursprungs invarianten Implementierung. Wolinski, Hinton und Pulay\supercite{wolinski1990efficient} implementierten 1990 eine deutlich effizientere Methode auf Hatree-Fock-Niveau, welche die beiden oben genannten Punkte zum ersten Mal erfüllen konnte. Dies wurde durch eine effizient abgeschätzte \glqq \textit{on the fly}\grqq{}-Berechnung der Vierzentren-Zweielektronen-Integrale sowie durch das Vermeiden der Transformation dieser Integrale erreicht. Kurz darauf folgte die Entwicklung von Methoden zur Berechnungen auf dem Level der \ac{mp2}\supercite{gauss1992calculation} sowie auf \ac{cc}\supercite{gauss1995gauge} Niveau von Gauss und Stanton. Für diese Methoden wurde eine hohe Genauigkeit berechneter $^1$H und $^{13}$C Verschiebungen für einen Testsatz mit kleinen organischen Molekülen demonstriert. Allerdings liegt der Rechenaufwand und Speicherbedarf deutlich über dem der Hartree-Fock-Rechnungen, so dass diese Methoden nicht auf größere Moleküle angewendet werden können. Eine deutlich effizientere Berechnung auf dem Niveau der \ac{dft} wurde etwa zur selben Zeit von Lee, Handy und Colwell\supercite{lee1995density} vorgestellt. 

\bigskip
Die im Rahmen der vorliegenden Arbeit durchgeführten Implementierungsarbeiten haben zwei wesentliche Ziele. Zum einen soll die Funktionalität des Moduls für die Berechnung chemischer Abschirmungskonstanten \texttt{mpshift}\supercite{haser1992direct,kollwitz1996direct} des Programmpaketes \textsc{TURBOMOLE}\supercite{ahlrichs1989electronic,TURBOMOLE,furche2014turbomole} erweitert werden. Zum anderen soll die Effizienz dieses Moduls gesteigert werden, um die Berechnung großer Systeme zu ermöglichen. Die erweiterte Funktionalität soll dabei die Berechnung von chemischen Abschirmungskonstanten für anionische Verbindungen, sowie für Verbindungen, welche schwere Elemente (Kernladung > 36) beinhalten, ermöglichen und Umgebungseffekte bei der Berechnung mit einbeziehen. Um dies zu ermöglichen ist es notwendig das \ac{cosmo}\supercite{klamt1993cosmo} und die sogenannten \acp{ecp} für die Berechnung der chemischen Abschirmkonstanten zu implementieren. Neben der \ac{nmr} Spektroskopie gewann auch die \ac{vcd} Spektroskopie in der näheren Vergangenheit deutlich an Popularität. Mit ihrer Hilfe und insbesondere durch den notwendigen Vergleich von gemessenen und berechneten Spektren, lassen sich die absoluten Konfigurationen in Molekülen bestimmen. Bei ihrer Berechnung wird ebenfalls die Response der Wellenfunktion des Moleküls auf ein externes Magnetfeld benötigt. Um die \ac{vcd} Spektren berechnen zu können, ist es daher notwendig die entsprechenden Gleichungen in das Programmpaket zu implementieren. 

Zur Berechnung der chemischen Abschirmungskonstanten oder \ac{vcd} Spektren in großen Molekülen soll des Weiteren die Effizienz des Moduls \texttt{mpshift} deutlich verbessert werden. Dafür ist es notwendig, die \ac{ri} Methode auf die nach den Komponenten des Magnetfeldes abgeleiteten Vierzentren-Zweielektronen-Integrale zu übertragen. Eine zusätzliche Steigerung der Effizienz soll durch die Adaption des \ac{marij} Verfahrens\supercite{sierka2003fast} erreicht werden. Die \ac{ri} Methode und die zusätzliche Beschleunigung durch die Multipolnäherung können bereits bei der Berechnung des Coulombbeitrages für die Wellenfunktion des elektronischen Grundzustandes verwendet werden und versprechen auch bei der Berechnung chemischer Abschirmungskonstanten einen deutlichen Effizienzgewinn. Durch das Anwenden dieser Näherungsverfahren wird der Hartree-Fock-Austausch unabhängig vom Coulombbeitrag berechnet. Da Ersterer schneller abfällt kann eine effizientere Integralabschätzung für die Berechnung dieses Beitrages implementiert werden. Weitere Beschleunigungen der Berechnungen sollen durch die Implementierung einer moderaten OpenMP-Parallelisierung sowie durch gezielte Optimierung des Programmcodes gewährleistet werden. 

Neben der Berechnung von \ac{nmr} und \ac{vcd} Spektren kann das Modul \texttt{mpshift} auch die nach den Komponenten des Magnetfeldes abgeleitete Elektronendichte zur Verfügung stellen. Diese wird beispielsweise von dem Programm GIMIC(acl{gimic})\supercite{juselius2004calculation,taubert2011calculation,fliegl2011gauge,sundholm2016calculations} zur Berechnung der Stromdichte benötigt. Des Weiteren lassen sich damit Ringströme in Molekülen berechnen, welche ein Maß für die Delokalisierung der Elektronen darstellen. Fließt der resultierende Gesamtringstrom im Uhrzeigersinn wird dieser als diatropisch bezeichnet und das Molekül als aromatisch klassifiziert. Typisch aromatische Verbindungen wie Benzol oder Porphyrin besitzen beispielsweise diatropische Gesamtringströme von etwa \unit[12]{nA/T} bzw. \unit[27]{nA/T}.\supercite{fliegl2012aromatic} Antiaromatische Verbindungen besitzen dahingegen einen paratropischen Gesamtringstrom, welcher gegen den Uhrzeigersinn fließt.

\bigskip
Schließlich sollen die im Rahmen der vorliegenden Arbeit implementierten Methoden auf chemische Fragestellungen angewendet werden. Die gesteigerte Effizienz ermöglicht die Untersuchung der magnetischen Eigenschaften von großen toroidalen Kohlenstoff Nanoröhren mit 1000 und mehr Atomen auf \ac{dft}-Niveau. Aufgrund ihrer einzigartigen Geometrie stellen diese Verbindungen sowohl auf experimenteller aber auch auf theoretischer Ebene ein aktuelles Forschungsgebiet dar. Ein fundamentales Verständnis ihrer Eigenschaften ist wichtig im Hinblick auf mögliche zukünftige Anwendungsgebiete. Durch die Berechnung von Ringströmen in diesen Systemen lassen sich Aussagen über die Delokalisierung der Elektronen darin treffen und Einflüsse struktureller Parameter beschreiben. 

Die Implementierung des \ac{cosmo} ermöglicht des Weiteren die Untersuchung der magnetischen Eigenschaften und die Berechnung von \ac{nmr} Spektren in anorganischen, anionischen Verbindungen. Mithilfe berechneter chemischer Verschiebungen werden die experimentell gemessenen \ac{nmr} Spektren von endohedralen Zinn-Antimon-Clusteranionen gedeutet und es erfolgt eine Zuordnung der Signale auf die atomaren Positionen. 

\bigskip
Die vorliegende Arbeit ist wie folgt aufgebaut. Kapitel \ref{theorie} beinhaltet die allgemeinen und für diese Arbeit notwendigen theoretischen Grundlagen. Neben einer kurzen Zusammenfassung des Hartree-Fock-Verfahrens und der \ac{dft} ist dort eine ausführliche Herleitung der notwendigen Gleichungen für die Berechnung der chemischen Abschirmungskonstanten auf diesen beiden Niveaus gegeben. Zusätzlich wird am Ende des Kapitels kurz auf die Berechnung der Stromdichte und Ringströme mit dem Programm \ac{gimic} eingegangen. Eine schematische Darstellung der Programmstruktur des Moduls \texttt{mpshift}, wie es vor den Änderungen die im Rahmen dieser Arbeit durchgeführt werden vorlag, ist in Kapitel \ref{programmstruktur} gezeigt. Dort wird die Funktion der wichtigsten Routinen zur Berechnung chemischer Abschirmungskonstanten erläutert. Die darauffolgenden Kapitel \ref{effizienz} und \ref{funktionalität} beschreiben die Implementierungen in das Programmpaket \textsc{TURBOMOLE} die im Rahmen dieser Arbeit durchgeführt werden. In den darin enthaltenen Unterkapiteln wird zunächst eine kurze Einführung in die jeweilige Theorie gegeben, bevor die eigentliche Implementierung im Detail erläutert wird. Sofern es von Relevanz ist, wird dort ebenfalls eine schematische Darstellung der Programmstruktur mit den neu implementierten oder modifizierten Routinen gezeigt. Ausgewählte Testrechnungen veranschaulichen die neu implementierten Funktionalitäten. In Kapitel \ref{genauigkeit} wird auf die Genauigkeit und die Effizienz der implementierten Näherungsverfahren zur Verkürzung der Rechenzeit eingegangen. Schließlich werden in Kapitel \ref{anwendungen} vier Anwendungsbeispiele gezeigt auf die die neuen Entwicklungen, welche durch diese Arbeit ermöglicht wurden, angewendet werden. Mit ihrer Hilfe sollen chemische Fragestellungen beantwortet werden. Eine abschließende Zusammenfassung dieser Arbeit erfolgt in Kapitel \ref{zusammenfassung} bevor in Kapitel \ref{ausblick} ein kurzer Ausblick auf eine mögliche Implementierung der semiempirischen Berechnung des Hartree-Fock-Austauschs für chemische Abschirmungskonstanten gegeben wird.