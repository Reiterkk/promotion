Das Modul \texttt{mpshift} umfasste bisher die Möglichkeit zur Berechnung von chemischen Abschirmungskonstanten in Molekülen ohne schwere Elemente (ab etwa $Z$=36) in der Gasphase. Diese Berechnungen konnten auf Hartree-Fock, \ac{dft} (für \ac{lda}- und \ac{gga}-Funktionale) und \ac{mp2} Nievau durchgeführt werden. Des Weiteren konnte die nach den Komponenten des Magnetfeldes abgeleitete Dichtematrix dem externen Programm \ac{gimic} zur Weiterverarbeitung bereit gestellt werden. Die folgenden Kapitel beschreiben die Erweiterung der Funktionalität des Moduls die im Rahmen dieser Arbeit implementiert wurden. Im Einzelnen sind dies die Berücksichtigung relativistischer Effekte auf Nachbaratome in Molekülen mit schweren Elementen durch relativistische \acp{ecp}, die Einbeziehung von Umgebungseffekten sowie die Bereitstellung der magnetischen Response zur Berechnung von \ac{vcd}-Spektren. Weiterhin wurde das Modul um die Möglichkeit ergänzt, \ac{mgga}-Funktionale für die Berechnung der Abschirmungskonstanten auf \ac{dft}-Niveau zu verwenden. Die eigentliche Implementierung dieses letzten Punktes erfolgte jedoch nicht von mir, sondern von Fabian Mack im Rahmen seiner Masterarbeit.\supercite{mack2017} 


\section{Berücksichtigung von Umgebungseffekten}
Isotrope chemische Verschiebungen werden üblicherweise in Lösung gemessen. Je nachdem wie stark das zu untersuchende Molekül mit dem Lösungsmittel wechselwirkt, hat das Lösungsmittel einen mehr oder weniger stark ausgeprägten Einfluss auf das gemessene Spektrum. Das \ac{cosmo}\supercite{klamt1993cosmo}, ein Kontinuumsmodell, ist in der Quantenchemie ein bewährtes Verfahren zur Berücksichtigung von Umgebungseffekten. Neben den Einflüssen des Lösungsmittels lassen sich damit auch für Ionen die Ladungen kompensieren, ohne die Gegenionen explizit mitrechnen zu müssen. Insbesondere hoch geladene Anionen lassen sich ohne eine solche Ladungskompensation nur schwer oder gar nicht berechnen. 

Neben \ac{cosmo} besteht auch die die Möglichkeit, Lösungsmittelmoleküle explizit in die Berechnung mit einzubeziehen. Soll eine größere Anzahl an Lösungsmitteln explizit betrachtet werden, um beispielsweise eine vollständige Solvatationshülle um das gelöste Molekül zu erhalten, so bieten sich \ac{md}-Simulationen an. Aus diesen Simulationen für ein bestimmtes Zeitintervall, lassen sich einzelne Schnappschüsse der Molekülkoordinaten extrahieren. Für diese einzelnen Koordinaten lassen sich dann \textit{ab initio} Rechnungen durchführen. Ein Vergleich solch unterschiedlicher Ansätze ist in Kapitel \ref{lömitest} für das Acetonmolekül in Wasser zu finden. 
	\subsection{Theorie}
	In einem Kontinuum-Lösungsmittel-Model (englisch \ac{csm}), wie dem \ac{cosmo}, wird das zu betrachtende Lösungsmittel durch seine Dielektrizitätskonstante $\varepsilon$ beschrieben. Das gelöste Molekül stellt dabei einen Hohlraum im dielektrischen Kontinuum dar und polarisiert das dielektrische Medium aufgrund seiner Ladungsverteilung. Zur Beschreibung der Reaktion des dielektrischen Mediums auf diese Polarisierung, werden auf der Oberfläche des durch das gelöste Molekül entstandenen Hohlraums sogenannte \textit{screening}-Ladungen generiert. In der Praxis wird also in einem bestimmten Abstand eine Hülle um das gelöste Molekül gelegt und auf der Oberfläche dieser Hülle befinden sich diese Ladungen. Beim \ac{cosmo} wird nun die Nebenbedingung eingeführt, dass das elektrostatische Potential auf der Oberfläche dieser Hülle verschwinden soll, was einem idealen Lösungsmittel mit unendlicher Dielektrizitätskonstante $\varepsilon=\infty$ entspricht. Das gesamte elektrostatische Potential $\vec{\phi}^{\textrm{Tot}}$ setzt sich aus dem Beitrag des gelösten Moleküls $\vec{\phi}^{\textrm{Mol}}$ und dem Beitrag der \textit{screening}-Ladungen $\boldsymbol{A}\vec{q}$. $\vec{\phi}^{\textrm{Mol}}$ beinhaltet dabei sowohl die Beiträge der Elektronen, als auch die der Kerne. Der Vektor $\vec{q}$ enthält die insgesamt $N_{\textrm{SL}}$ \textit{screening}-Ladungen und die Matrix $\boldsymbol{A}$ beinhaltet die Coulombwechselwirkung der \textit{screening}-Ladungen untereinander. Mit der Bedingung des verschwindenden elektrostatischen Potentials folgt daher
	
	\begin{equation}
	\vec{\phi}^{\textrm{Tot}}=\vec{\phi}^{\textrm{Mol}}+\boldsymbol{A}\vec{q}=0,
	\end{equation}
	
wodurch sich die \textit{screening}-Ladungen definieren lassen
	\begin{equation}
	\vec{q}=\boldsymbol{A}^{-1}\vec{\phi}^{\textrm{Mol}}
	\end{equation}
Um nun Lösungsmittel mit unterschiedlichen Dielektrizitätskonstanten betrachten zu können, wird ein Skalierungsfaktor $f(\varepsilon)$ eingeführt
	\begin{equation}
	f(\varepsilon)=\frac{\varepsilon-1}{\varepsilon+\frac{1}{2}}
	\end{equation}
und damit lassen sich die entsprechenden \textit{screening}-Ladungen $\vec{q}(\varepsilon)$ erhalten
	\begin{equation}
	\vec{q}(\varepsilon)=\vec{q}(\varepsilon=\infty)f(\varepsilon).
	\end{equation}
Die Abweichungen aufgrund der hier gewählten Nebenbedingung des verschwindenden elektrostatischen Potentials im Vergleich zu den eigentlich viel komplexeren Nebenbedingungen ist sehr gering.\supercite{klamt1993cosmo} Dies gilt insbesondere für Lösungsmittel mit großen Dielektrizitätskonstanten, wie beispielsweise Wasser.

Bei der Berechnung chemischer Abschirmungskonstanten müssen die erzeugten Punktladungen auf der Oberfläche der Hülle um das gelöste Molekül ebenfalls berücksichtigt werden. Die \textit{screening}-Ladungen ergeben einen zusätzlichen Energiebeitrag $E^{\textrm{SM}}$ für den Einelektronenteil\supercite{cammi1999nuclear}

	\begin{equation}\label{eq:esm}
	E^{\textrm{SM}}=\sum_{l}^{N_{\textrm{SL}}}\int\frac{q_l\rho(\vec{r})}{\vert\vec{t}_l-\vec{r}\vert}\diff\vec{r}=\sum_{\mu\nu}D_{\mu\nu}\underbrace{\sum_l^{N_{\textrm{SL}}}\int\frac{q_l\chi_\mu\chi_\nu}{\vert\vec{t}_l-\vec{r}\vert}\diff\vec{r}}_{V_{\mu\nu}^{\textrm{SM}}}.
	\end{equation}
mit den Positionen $\vec{t}_l$ der \textit{screening}-Ladungen. Für die chemischen Abschirmungskonstanten wird daher die Ableitung von Gleichung (\ref{eq:esm}) nach den Komponenten des Magnetfeldes benötigt. Dies führt schließlich zu

	\begin{equation}\label{eq:esmdb}
	E^{\textrm{SM},B_\beta}=\sum_{\mu\nu}D_{\mu\nu}\underbrace{\frac{\iu}{2c}\sum_l^{N_{\textrm{SL}}}\int\left(\vec{R}_{\mu\nu}\times\vec{r}\right)\frac{q_l\chi_\mu^{\vec{B}=0}\chi_\nu^{\vec{B}=0}}{\vert\vec{t}_l-\vec{r}\vert}\diff\vec{r}}_{V_{\mu\nu}^{\textrm{SM},B_\beta}}.
	\end{equation}
	
	\subsection{Implementierung}
	Bei genauer Betrachtung von Gleichung (\ref{eq:esmdb}) fällt auf, dass die $V_{\mu\nu}^{\textrm{SM},B_\beta}$ die selbe Form haben wie die nach den Komponenten des Magnetfeldes abgeleitete Kern-Elektron-Wechselwirkung $V_{\textrm{Ke}\mu\nu}^{B_\beta}$. Zur Implementierung der \ac{cosmo} Beiträge bei der Berechnung chemischer Abschirmungskonstanten können daher die bereits bestehenden Routinen zur Berechnung des letztgenannten Beitrages modifiziert werden. Es ist dabei lediglich darauf zu achten, dass anstelle der Kernladungen die \textit{screening}-Ladungen und anstelle der Kernpositionen die Positionen der \textit{screening}-Ladungen an die entsprechende Routine übergeben werden. In der bei der Verwendung von \ac{cosmo} zur Berechnung von chemischen Abschirmungskonstanten wird daher die Routine \texttt{vsints} ein weiteres Mal von der Routine \texttt{csplop} mit den entsprechenden Feldern gerufen. 
	
Grundsätzlich lassen sich auf diese Weise die Beiträge von beliebige Punktladungen berechnen. Anstelle von \ac{cosmo}, was Punktladungen auf einer Hülle um das gelöste Molekül generiert, besteht daher auch die Möglichkeit die elektrostatische Wechselwirkung expliziter Lösungsmittelmoleküle durch Punktladungen zu ersetzen. Dafür können einzelne Momentaufnahmen aus \ac{md}-Simulationen verwendet werden. An die Positionen der Atome der Lösungsmittel werden Punktladungen gesetzt. Der Betrag der jeweiligen Ladung kann beispielsweise durch eine Populationsanalyse wie Mulliken\supercite{mulliken1955electronic} oder \ac{npa}\supercite{reed1985natural} bzw. durch einen \ac{esp}-Fit\supercite{singh1984approach} für das isolierte Lösungsmittelmolekül bestimmt werden. Die entsprechenden Koordinaten und Ladungen werden schließlich wieder an die Routine \texttt{vsints} übergeben. Eine Alternative zu den Punktladungen sind gaußförmig verschmierte Ladungen. Die dafür notwendigen Dreizentrenintegrale wurden bereits für die \ac{ri}-Näherung benötigt und können wiederverwendet werden. Um dies zu gewährleisten wird die Routine \texttt{csonei} um einen Aufruf der Routine \texttt{cslp3} erweitert. Im Vergleich zu punktförmigen Ladungen lässt sich dadurch eine etwas weichere Ladungsverteilung generieren. 

Die unterschiedlichen Möglichkeiten zur Einbeziehung von Lösungsmitteleffekten wurden am Beispiel des Acetonmoleküls in Wasser untersucht und ist im folgenden Kapitel erläutert.
	\subsection{Testrechnungen}\label{lömitest}
	
\section{Skalar-relativistische Effekte durch effektive Kernpotentiale}
Für schwere Atome nimmt mit steigender Kernladungszahl der Einfluss relativistischer Effekte zu. Diese relativistischen Einflüsse haben ihren Ursprung in Kernnähe schwerer Atome. Sie übertragen sich jedoch auch auf die Valenzschalen der entsprechenden Atome und haben damit auch einen Einfluss auf die chemische Verschiebung an benachbarten Atomen. Die vollrelativistische Berechnung im Rahmen vier- oder zweikomponentiger Methoden (wie beispielsweise das X2C-Verfahren) ist sehr aufwändig. Eine alternative Berücksichtigung skalarrelativistischer Effekte ist durch die Verwendung von sogenannten \acp{ecp}\supercite{cundari1996effective,frenking2007pseudopotential} gegeben. Hierbei werden die Elektronen in den Rumpforbitalen durch ein entsprechend gefittetes Potential beschrieben und nur die Valenzelektronen explizit betrachtet. Aufgrund der fehlenden kernnahen Elektronen haben chemische Abschirmungskonstanten, welche für Atome mit einem \ac{ecp} berechnet wurden, keine physikalische Bedeutung. Die Rumpfelektronen liefern den größten Beitrag zur Abschirmung und daher wird diese stark unterschätzt. Wird der durch das \ac{ecp} beschriebene Bereich nicht zu groß gewählt, d.h. sogenannte \textit{small core} \acp{ecp} verwendet, dann kann jedoch bei der Berechnung relativer chemischer Verschiebungen davon ausgegangen werden, dass sich der fehlende kernnahe Beitrag aufhebt.\supercite{van2012use} In diesem Zusammenhang untersuchten Moore und Healy\supercite{moore1995ab} die Abhängigkeit der Titan Abschirmung in Titan-Tetrahalogeniden von allelektornen Basissätzen sowie \acp{ecp} und kamen zu dem Schluss, dass die absoluten Abschirmungen stark von der gewählten Basis abhängen, die relativen chemischen Verschiebungen jedoch weitgehend unabhängig davon sind. Bagno und Bonchio konnten ebenfalls zeigen, dass sich die chemischen Verschiebungen, berechnet mit \acp{ecp}, von Wolfram\supercite{bagno2000effective} und Ruthenium\supercite{bagno2002dft} gut mit experimentell gemessenen Daten korrelieren lassen. Problematisch ist jedoch, dass die so erhaltenen chemischen Verschiebungen zunächst an experimentell gemessene Verschiebungen gefittet werden müssen, um eine Korrelation herzustellen. Erst damit lassen sich Aussagen über die chemische Verschiebung in unbekannten Verbindungen treffen. Für die Berechnung von chemischen Abschirmungskonstanten an benachbarten Atomen schwerer Atome können die \acp{ecp} jedoch problemlos verwendet werden.
	\subsection{Theorie}
	Eine eichinvariante Implementierung für \acp{ecp} wurde von van Wüllen\supercite{van2012use} vorgestellt und wichtigsten darin abgeleiteten Gleichungen sollen an dieser Stelle wiedergegeben werden. 
	
	Der Einelektronenhamiltonoperator im Magnetfeld $\vec{B}$ setzt sich aus dem ungestörten Hamiltonoperator $\hat{h}^0$ in Abwesenheit des Magnetfeldes sowie weiteren Termen linear, quadratisch usw. in $\vec{B}$ zusammen. Für die chemische Verschiebung werden jedoch nur Terme linear in $\vec{B}$ benötigt, daher ist $\hat{h}$ hier
	\begin{equation}\label{eq:heinel}
	\begin{aligned}
	\hat{h}=&\hat{h}^0+\hat{h}^{10}\\
	\hat{h}^0=&\frac{1}{2}\vec{p}^{\,2}+\hat{V}_{\textrm{Ke}}\\
	\hat{h}^{10}=&\frac{1}{2c}\left(\left(\vec{r}-\vec{R}_E\right)\times\vec{p}\right)\cdot\vec{B}.
	\end{aligned}
	\end{equation}
	Wie bereits zuvor in Kapitel \ref{theo:nmr} erwähnt, kannn der Eichursprung willkürlich gewählt werden und der Hamiltonoperator ist genau dann eichinvariant, wenn für unterschiedliche Eichursprünge die selben Werte magnetischer Eigenschaften berechnet werden. Dies ist dann erfüllt, wenn der Hamiltonoperator $\hat{\tilde{h}}$ mit dem Eichursprung $\vec{R}_{\tilde{E}}$ aus $\hat{h}$ durch eine unitäre Transformation der Form
	\begin{equation}\label{eq:trans}
	\hat{\tilde{h}}=\exp\left(\iu\Lambda_{\tilde{E}}\right)\hat{h}\exp\left(-\iu\Lambda_{\tilde{E}}\right),
	\end{equation}
	mit 
	\begin{equation}
	\begin{aligned}
	\Lambda_{\tilde{E}}=&\frac{1}{2c}\left(\left(\vec{R}_{\tilde{E}}-\vec{R}_E\right)\times\vec{r}\right)\cdot\vec{B}\\
	=&\frac{1}{2c}\left(\vec{B}\times\left(\vec{R}_{\tilde{E}}-\vec{R}_E\right)\right)\cdot\vec{r}
	\end{aligned}
	\end{equation}
	erhalten werden kann. Zur Bestimmung von $\hat{\tilde{h}}$ wird dessen Wirkung auf eine Funktion $f(\vec{r})$ betrachtet. Aus der unitären Transformation folgt
	
	\begin{equation}\label{eq:transh}
	\begin{aligned}
	\hat{\tilde{h}}f(\vec{r})=&\exp\left(\iu\Lambda_{\tilde{E}}\right)\hat{h}\exp\left(-\iu\Lambda_{\tilde{E}}\right)f(\vec{r})\\
	=&\exp\left(\iu\Lambda_{\tilde{E}}\right)\hat{h}^0\exp\left(-\iu\Lambda_{\tilde{E}}\right)f(\vec{r})+\exp\left(\iu\Lambda_{\tilde{E}}\right)\hat{h}^{10}\exp\left(-\iu\Lambda_{\tilde{E}}\right)f(\vec{r})
	\end{aligned}
	\end{equation}
	Für den ersten Term auf der rechten Seite von Gleichung (\ref{eq:transh}) ergibt sich
	\begin{equation}\label{eq:transh1}
	\begin{aligned}
	&\exp\left(\iu\Lambda_{\tilde{E}}\right)\hat{h}^0\exp\left(-\iu\Lambda_{\tilde{E}}\right)f(\vec{r})\\
	=&\exp\left(\iu\Lambda_{\tilde{E}}\right)\frac{-\iu}{2}\vec{p}\left[\frac{-\iu}{2c}\left(\vec{B}\times\left(\vec{R}_{\tilde{E}}-\vec{R}_E\right)\right)\exp\left(-\iu\Lambda_{\tilde{E}}\right)f(\vec{r})\right]\\
	&+\exp\left(\iu\Lambda_{\tilde{E}}\right)\frac{1}{2}\vec{p}\left[\exp\left(-\iu\Lambda_{\tilde{E}}\right)\vec{p}\left(f(\vec{r})\right)\right]+\hat{V}_{\textrm{Ke}}f(\vec{r})\\
	=&-\frac{1}{4c}\left(\vec{B}\times\left(\vec{R}_{\tilde{E}}-\vec{R}_E\right)\right)\vec{p}\left(f(\vec{r})\right)
	-\frac{1}{4c}\left(\vec{B}\times\left(\vec{R}_{\tilde{E}}-\vec{R}_E\right)\right)\vec{p}\left(f(\vec{r})\right)\\
	&+\frac{1}{2}\vec{p}^{\,2}f(\vec{r})+\hat{V}_{\textrm{Ke}}f(\vec{r})\\
	=&\frac{-1}{2c}\left(\vec{B}\times\left(\vec{R}_{\tilde{E}}-\vec{R}_E\right)\right)\vec{p}\left(f(\vec{r})\right)+\hat{h}^0f(\vec{r}),
	\end{aligned}
	\end{equation}
	und für den zweiten Term
	\begin{equation}\label{eq:transh2}
	\begin{aligned}
	\exp\left(\iu\Lambda_{\tilde{E}}\right)\hat{h}^{10}\exp\left(-\iu\Lambda_{\tilde{E}}\right)f(\vec{r})=&\frac{1}{2c}\left(\vec{B}\times\left(\vec{r}-\vec{R}_E\right)\right)\cdot\vec{p}\left(f(\vec{r})\right)\\
	&+\frac{1}{2c}\left(\vec{B}\times\left(\vec{r}-\vec{R}_E\right)\right)\cdot\frac{-1}{2c}\left(\vec{B}\times\left(\vec{R}_{\tilde{E}}-\vec{R}_E\right)\right)f(\vec{r})\\
	=&\hat{h}^{10}f(\vec{r})+\mathcal{O}(\vec{B}^{\,2}).
	\end{aligned}
	\end{equation}
	Für den Hamiltonoperator $\hat{\tilde{h}}$ folgt aus den Gleichungen (\ref{eq:transh1}) und (\ref{eq:transh2})
	\begin{equation}
	\begin{aligned}
	\hat{\tilde{h}}=&\hat{h}^0+\hat{h}^{10}-\frac{1}{2c}\left(\vec{B}\times\left(\vec{R}_{\tilde{E}}-\vec{R}_E\right)\right)\vec{p}\\
	=&\hat{h}^0+\frac{1}{2c}\left(\left(\vec{r}-\vec{R}_{\tilde{E}}\right)\times\vec{p}\right)\cdot\vec{B}
	\end{aligned}
	\end{equation}
	
	wobei die in $\vec{B}$ quadratischen Terme hier erneut weggelassen wurden. Der Eichursprung wurde durch die Transformation also von $\vec{R}_E$ auf $\vec{R}_{\tilde{E}}$ verschoben. Für diese Herleitung wurde davon ausgegangen, dass die Kern-Elektron-Wechselwirkung ein lokales Potential ist und daher nicht mit $\vec{r}$ kommutiert. Dies ist nur gültig, solange keine \acp{ecp} verwendet werden. Im letzteren Fall ist der Einelektronenhamiltonoperator aus Gleichung (\ref{eq:heinel}) durch
	
	\begin{equation}
	\hat{h}^0=\frac{1}{2}\vec{p}^{\,2}-\sum_K \left(\frac{Z_K^{\textrm{eff}}}{\vec{r}_K}+\hat{V}^{\textrm{ECP},K}\right)
	\end{equation}
	gegeben. Die Valenzelektronen, die nicht durch das \ac{ecp} beschrieben werden, erfahren dann nur noch eine verminderte effektive Kernladung $Z_K^{\textrm{eff}}$. Für die \acp{ecp} ist das Potential durch eine Summe atomarer Beiträge geben und diese haben die Form\supercite{mcmurchie1981calculation,cao2010relativistic}
	\begin{equation}\label{eq:ecpotential}
	\hat{V}^{\textrm{ECP},K}=\sum_{l=0}^{\infty}\hat{V}^{\textrm{ECP},K}_l\hat{P}_l^K,
	\end{equation}
	mit dem Projektionsoperator
	\begin{equation}
	\hat{P}_l^K=\sum_{m=-l}^l\vert lm\rangle\langle lm\vert
	\end{equation}
	und den Kugelflächenfunktionen $\vert lm\rangle$. Für $l\geq L$ unterscheiden sich die $\hat{V}^{\textrm{ECP},K}_l$ kaum mehr, wobei $L-1$ die größte, in den Kernorbitalen auftretende Drehimpulsquantenzahl ist. Mit der Annahme $\hat{V}^{\textrm{ECP},K}_l=\hat{V}^{\textrm{ECP},K}_L$ für $l\geq L$ folgt schließlich\supercite{kahn1972ab}
	\begin{equation}
	\hat{V}^{\textrm{ECP},K}=\hat{V}^{\textrm{ECP},K}_{L}+\sum_{l=0}^{L-1}\sum_{m=-l}^l lm\rangle\left[\hat{V}^{\textrm{ECP},K}_{l}-\hat{V}^{\textrm{ECP},K}_{L}\right]\langle lm\vert .
	\end{equation}
	Die $\hat{V}^{\textrm{ECP},K}_{m}$ lassen sich nun durch eine Linearkombination von Gaußfunktionen multipliziert mit Potenzen von $\vec{r}$ ausdrücken\supercite{kahn1972ab}
	\begin{equation}
	\hat{V}^{\textrm{ECP},K}_{m}=\sum_jd_{jm}\vec{r}_K^{\,nj}e^{-\zeta_j\vec{r}_K^{\,2}}, \qquad \textrm{für } m=l,L,
	\end{equation}
	wobei die Exponenten $\zeta_j$ und die Koeffizienten $d_{jm}$ an sehr genaue Rechnungen gefittet werden. Der Drehimpuls-Projektionsoperator $\hat{P}_l^K$ in $\hat{V}^{\textrm{ECP},K}$ führt dazu, dass die \acp{ecp} nicht mehr mit $\vec{r}$ und damit mit $\Lambda$ kommutieren. 
	
	Aus dem magnetfeldabhängigen Hamiltonoperator für ein Einzelnes Atom $K$ mit dem Eichursprung $\vec{R}_K$ und der Transformation aus Gleichung (\ref{eq:trans}) lässt sich nun der magnetfeldabhängige \ac{ecp} Hamiltonoperator für eine beliebige Wahl des Eichursprungs ableiten. Es folgt
	
	\begin{equation}
	\begin{aligned}
	\hat{h}_{\textrm{ECP}}=&\exp\left(\iu\Lambda_K\right)\hat{h}_K\exp\left(-\iu\Lambda_K\right)\\
	=&\hat{h}^0+\left(\hat{h}^{10}+\iu\left[\hat{V}^{\textrm{ECP},K},\Lambda_K\right]\right)+\mathcal{O}(\vec{B}^{\,2})+\dotsc,
	\end{aligned}
	\end{equation}
	mit
	\begin{equation}
	\Lambda_K=\frac{1}{2c}\left(\left(\vec{R}_K-\vec{R}_E\right)\times\vec{r}\right)\cdot\vec{B}.
	\end{equation}
	Für Moleküle muss zusätzlich über die Beiträge aller Atome summiert werden
	\begin{equation}\label{eq:hecp}
	\begin{aligned}
	&\hat{h}_{\textrm{ECP}}=\hat{h}^0+\hat{h}_{\textrm{ECP}}^{10}\\
	&\hat{h}_{\textrm{ECP}}^{10}=\hat{h}^{10}+\iu\sum_K\left[\hat{V}^{\textrm{ECP},K},\Lambda_K\right].
	\end{aligned}
	\end{equation}	 
	Die zusätzlichen Integrale, die durch den Kommutator in Gleichung (\ref{eq:hecp}) auftreten lassen sich durch Entwicklung des Integrals $\langle\chi_\mu\vert\hat{h}_{\textrm{ECP}}\vert\chi_\nu\rangle$ und Angabe der Terme linear in $\vec{B}$ erhalten. Da sowohl die Basisfunktionen als auch der Hamiltonoperator vom Magnetfeld abhängen, ergeben sich für die Terme linear in $\vec{B}$
	
	\begin{equation}\label{eq:linearinb}
	\begin{aligned}
	\langle &\chi_\mu^{\vec{B}=0}\vert\hat{h}_{\textrm{ECP}}^{10}+\iu\Lambda_\mu\hat{h}^0-\iu\hat{h}^0\Lambda_\nu\vert\chi_\nu^{\vec{B}=0}\rangle\\
	&=\langle \chi_\mu^{\vec{B}=0}\vert\hat{h}_{\textrm{ECP}}^{10}\vert\chi_\nu^{\vec{B}=0}\rangle+\iu\langle\chi_\mu^{\vec{B}=0}\vert\left(\Lambda_\mu -\Lambda_\nu\right)\hat{h}^0\vert\chi_\nu^{\vec{B}=0}\rangle-\iu\langle\chi_\mu^{\vec{B}=0}\vert\left[\hat{h}^0 ,\Lambda_\nu\right]\vert\chi_\nu^{\vec{B}=0}\rangle\\
    &=\langle \chi_\mu^{\vec{B}=0}\vert\frac{1}{2c}\left(\left(\vec{r}-\vec{R}_\nu\right)\times\vec{p}\right)\cdot\vec{B}\vert\chi_\nu^{\vec{B}=0}\rangle
    +\iu\langle\chi_\mu^{\vec{B}=0}\vert\left(\Lambda_\mu -\Lambda_\nu\right)\hat{h}^0\vert\chi_\nu^{\vec{B}=0}\rangle\\
    &\quad+\iu\sum_K\langle \chi_\mu^{\vec{B}=0}\vert\left[\hat{V}^{\textrm{ECP},K},\Lambda_K-\Lambda_\nu\right]\vert\chi_\nu^{\vec{B}=0}\rangle,
	\end{aligned}
	\end{equation}
	
	wobei der Kommutator
	
	\begin{equation}
	\iu\left[\hat{h}^0 ,\Lambda_\nu\right]=\frac{1}{2c}\left(\left(\vec{R}_\nu-\vec{R}_E\right)\times\vec{p}\right)\cdot\vec{B}+\iu\left[\hat{V}^{\textrm{ECP},K} ,\Lambda_\nu\right]
	\end{equation}
	ausgenutzt wurde. Der letzte Term in Gleichung (\ref{eq:linearinb}) ist ein zusätzlicher Term, der im Rahmen des von van Wüllen vorgeschlagenen \ac{ecp} \ac{giao} Formalismus auftritt, alle anderen Terme sind bereits bekannt. Anhand der Terme ist zu erkennen, dass all diese Ausdrücke nun nicht mehr von dem Eichursprung abhängen, da dieser nur noch in den Differenzen von $\Lambda_K$ und $\Lambda_\nu$ vorkommt. Werden nun alle Terme die aufgrund der \acp{ecp} entstehen kombiniert, dann wird der letztendlich zu implementierende Ausdruck erhalten
	
	\begin{equation}\label{eq:vecpmunu}
	\begin{aligned}
	V_{\mu\nu}^{\textrm{ECP}}=&\iu\sum_K\langle\chi_\mu^{\vec{B}=0}\vert\left(\Lambda_\mu -\Lambda_\nu\right)\hat{V}^{\textrm{ECP},K}\vert\chi_\nu^{\vec{B}=0}\rangle\\
	&+\iu\sum_K\langle \chi_\mu^{\vec{B}=0}\vert\left[\hat{V}^{\textrm{ECP},K},\Lambda_K-\Lambda_\nu\right]\vert\chi_\nu^{\vec{B}=0}\rangle\\
	=&\iu\sum_K\langle\chi_\mu^{\vec{B}=0}\vert\left(\Lambda_\mu -\Lambda_K\right)\hat{V}^{\textrm{ECP},K}-\hat{V}^{\textrm{ECP},K}\left(\Lambda_\nu -\Lambda_K\right)\vert\chi_\nu^{\vec{B}=0}\rangle .
	\end{aligned}
	\end{equation}
	
	\subsection{Implementierung}
	Zur Implementierung der \ac{ecp} Beiträge für die Berechnung chemischer Abschirmungskonstanten muss Gleichung (\ref{eq:vecpmunu}) nach den Komponenten des Magnetfeldes abgeleitet werden. Die Ableitung von Gleichung (\ref{eq:vecpmunu}) nach der $x$-Komponente ergibt
	\begin{equation}\label{eq:vecpdbx}
	\begin{aligned}
	V_{\mu\nu}^{\textrm{ECP},B_x}=&\left.\frac{\partial V_{\mu\nu}^{\textrm{ECP}}}{\partial B_x}\right\vert_{\vec{B}=0}\\
	=&\frac{\iu}{2c}\sum_K\left[\left\langle\chi_\mu^{\vec{B}=0}\left\vert\left(\left(R_{\mu y}-R_{Ky}\right)z-(\left(R_{\mu z}-R_{Kz}\right)y\right)\hat{V}^{\textrm{ECP},K}\right.\right.\right.\\
	&\left.\left.\left.-\hat{V}^{\textrm{ECP},K}\left(\left(R_{\nu y}-R_{Ky}\right)z-(\left(R_{\nu z}-R_{Kz}\right)y\right)\right\vert\chi_\nu^{\vec{B}=0}\right\rangle\right]\\
	=&\frac{\iu}{2c}\sum_K\left[\left(R_{\mu y}-R_{Ky}\right)\left\langle\chi_\mu^{\vec{B}=0}\left\vert z\hat{V}^{\textrm{ECP},K}\right\vert\chi_\nu^{\vec{B}=0}\right\rangle\right.\\
	&-\left(R_{\mu z}-R_{Kz}\right)\left\langle\chi_\mu^{\vec{B}=0}\left\vert y\hat{V}^{\textrm{ECP},K}\right\vert\chi_\nu^{\vec{B}=0}\right\rangle\\
	&-\left(R_{\nu y}-R_{Ky}\right)\left\langle\chi_\mu^{\vec{B}=0}\left\vert \hat{V}^{\textrm{ECP},K} z\right\vert\chi_\nu^{\vec{B}=0}\right\rangle\\
	&\left.+\left(R_{\nu z}-R_{Kz}\right)\left\langle\chi_\mu^{\vec{B}=0}\left\vert \hat{V}^{\textrm{ECP},K} y\right\vert\chi_\nu^{\vec{B}=0}\right\rangle\right]
	\end{aligned}
	\end{equation}
	und die analogen Ausdrücke für die Ableitungen nach den $y$- und $z$- Komponenten des Magnetfeldes sind 
	\begin{equation}\label{eq:vecpdby}
	\begin{aligned}
	V_{\mu\nu}^{\textrm{ECP},B_y}=&\frac{\iu}{2c}\sum_K\left[\left(R_{\mu z}-R_{Kz}\right)\left\langle\chi_\mu^{\vec{B}=0}\left\vert x\hat{V}^{\textrm{ECP},K}\right\vert\chi_\nu^{\vec{B}=0}\right\rangle\right.\\
	&-\left(R_{\mu x}-R_{Kx}\right)\left\langle\chi_\mu^{\vec{B}=0}\left\vert z\hat{V}^{\textrm{ECP},K}\right\vert\chi_\nu^{\vec{B}=0}\right\rangle\\
	&-\left(R_{\nu z}-R_{Kz}\right)\left\langle\chi_\mu^{\vec{B}=0}\left\vert \hat{V}^{\textrm{ECP},K}x\right\vert\chi_\nu^{\vec{B}=0}\right\rangle\\
	&\left.+\left(R_{\nu x}-R_{Kx}\right)\left\langle\chi_\mu^{\vec{B}=0}\left\vert \hat{V}^{\textrm{ECP},K} z\right\vert\chi_\nu^{\vec{B}=0}\right\rangle\right]
	\end{aligned}
	\end{equation}
	
	\begin{equation}\label{eq:vecpdbz}
	\begin{aligned}
	V_{\mu\nu}^{\textrm{ECP},B_z}=&\frac{\iu}{2c}\sum_K\left[\left(R_{\mu x}-R_{Kx}\right)\left\langle\chi_\mu^{\vec{B}=0}\left\vert y\hat{V}^{\textrm{ECP},K}\right\vert\chi_\nu^{\vec{B}=0}\right\rangle\right.\\
	&-\left(R_{\mu y}-R_{Ky}\right)\left\langle\chi_\mu^{\vec{B}=0}\left\vert x\hat{V}^{\textrm{ECP},K}\right\vert\chi_\nu^{\vec{B}=0}\right\rangle\\
	&-\left(R_{\nu x}-R_{Kx}\right)\left\langle\chi_\mu^{\vec{B}=0}\left\vert \hat{V}^{\textrm{ECP},K}y\right\vert\chi_\nu^{\vec{B}=0}\right\rangle\\
	&\left.+\left(R_{\nu y}-R_{Ky}\right)\left\langle\chi_\mu^{\vec{B}=0}\left\vert \hat{V}^{\textrm{ECP},K} x\right\vert\chi_\nu^{\vec{B}=0}\right\rangle\right].
	\end{aligned}
	\end{equation} 
	Erneut wird an dieser Stelle die Beziehung $\vec{r}=\vec{R}_\mu+\vec{r}_\mu$ ausgenutzt um die Integrale in den Gleichungen (\ref{eq:vecpdbx})-(\ref{eq:vecpdbz}) umzuschreiben. Beispielsweise ist 
	\begin{equation}\label{eq:ecpintegral}
	\left\langle\chi_\mu^{\vec{B}=0}\left\vert z\hat{V}^K\right\vert\chi_\nu^{\vec{B}=0}\right\rangle=R_{\mu z}\left\langle\chi_\mu^{\vec{B}=0}\left\vert \hat{V}^K\right\vert\chi_\nu^{\vec{B}=0}\right\rangle+\left\langle\chi_\mu^{\vec{B}=0}\left\vert z_\mu\hat{V}^K\right\vert\chi_\nu^{\vec{B}=0}\right\rangle.
	\end{equation}
	Das erste Integral auf der rechten Seite von Gleichung (\ref{eq:ecpintegral}) ist ein Standard \ac{ecp} Integral. Beim zweiten Integral wurde die $z$-Komponente der Drehimpulsquantenzahl für die Basisfunktion $\chi_mu^{\vec{B}=0}$ um eins erhöht, da
	\begin{equation}
	\chi_\mu^{\vec{B}=0}z_\mu=x_\mu^ly_\mu^mx_\mu^{n+1}e^{-\zeta\vec{r}^{\, 2}_\mu}.
	\end{equation}
	Diese Art von Integralen werden mit einem anderen Vorfaktor auch für die Berechnung von kartesischen \ac{ecp} Gradienten benötigt. Für die \ac{ecp} Beiträge zu den chemischen Abschirmungskonstanten können also die Standard \ac{ecp} Integral- und \ac{ecp} Gradientenroutinen modifiziert werden. 
	
	\subsection{Testrechnungen}
	
\section{Berechnung von Vibrational Circular Dichroism Spektren}
	\subsection{Theorie}
	Die im Experiment gemessenen Intensitäten der \ac{vcd}-Spektroskopie $I_n$ sind proportional zu den in quantenchemischen Rechnungen zugänglichen Rotationsstärken $R_n$. Letztere werden aus dem Skalaprodukt vom \textcolor{myred}{elektrischen/\-elektronischen} und vom magnetischen Übergangsdipolmoment, $\vec{\mu}_n^{\,\text{el}}$ und $\vec{\mu}_n^{\,\text{mag}}$,  erhalten. Somit ergibt sich die \ac{vcd}-Intensität
	\begin{equation}
	  I_n\approx R_n = Im(\vec{\mu}_n^{\,\text{el}}\cdot\vec{\mu}_n^{\text{\,mag}})
	\end{equation}
	für einen Übergang aus dem Schwingungsgrundzustand in den angeregten Schwingungszustand $n$.\supercite{stephens1985theory,stephens1985vibrational} Im Rahmen der harmonishen Näherung sind das \textcolor{myred}{elektrische/\-elektronische} und das magnetische Übergangsdipolmoment gegeben durch \textcolor{myred}{(Zitat?)}
	
	\begin{equation}
	  (\mu_n^{\text{el}})_\beta=\sqrt{\frac{\hbar}{\omega_n}}\sum_{K\alpha}P_{\alpha\beta}^K S_{K\alpha,n}
	\end{equation}
	\begin{equation}
	  (\mu_n^{\text{mag}})_\beta=-\sqrt{2\hbar^3\omega_n}\sum_{K\alpha}M_{\alpha\beta}^KS_{K\alpha,n}.
	\end{equation}
	Hierbei ist $I$ die Zählvariable für die Atomkerne, $\alpha$ und $\beta$ beschreiben kartesische Koordinaten, $\omega_n$ ist die Schwingungsfrequenz der $n$-ten Schwingung und $S_{I\alpha,n}$ ist die Transformationsmatrix von kartesichen zu Normalkoordinaten. Sowohl der sogenannte \ac{apt} (Gleichung (\ref{eq:apt}) als auch der sogenannte \ac{aat} (Gleichung (\ref{eq:aat})) lassen sich in einen elektronischen und einen Kernbeitrag aufteilen
	\begin{equation}\label{eq:apt}
	  P^K_{\alpha\beta}=E^K_{\alpha\beta}+N^K_{\alpha\beta}
	\end{equation}
	\begin{equation}\label{eq:aat}
   	  M^K_{\alpha\beta}=I^K_{\alpha\beta}+J^K_{\alpha\beta}.
	\end{equation}
	Die Berechnung der Kernbeiträge 
	\begin{equation}
	  N^K_{\alpha\beta}=eZ_K\delta_{\alpha\beta}
	\end{equation}
	\begin{equation}
	  J^K_{\alpha\beta}=\iu\frac{eZ_K}{4\hbar c}\sum_K^{N_K}\varepsilon_{\alpha\beta\gamma}R^0_{K\gamma}
	\end{equation}
	ist trivial. $e$ ist die Elementarladung, $Z_K$ ist die Ladung des Kerns $K$ und $\delta_{\alpha\beta}$ ist das Kroneckerdelta. $c$ ist die Lichtgeschwindigkeit und $\varepsilon_{\alpha\beta\gamma}$ ist der Levi-Civita-Permutationstensor. Die Position des Kerns $K$ ist durch $R^0_{K\gamma}$ gegeben, wobei $\gamma$ für eine der drei kartesischen Raumkoordinaten steht und die hochgestellte $0$ symbolisiert die Auswertung in der Gleichgewichtsgeometrie.   
	
    Zur Berechnung der elektronischen Beiträge 
    \begin{equation}
      E^K_{\alpha\beta}=\left(\sum_{i=1}^{N_{\text{occ}}}\frac{\partial \langle\phi_i\vert r_\beta\vert\phi_i\rangle}{\partial R_{K\alpha}}\right)_{\vec{R}^0}
    \end{equation}
    \begin{equation}\label{eq:vcdaatel}
      I^K_{\alpha\beta}=\sum_{i=1}^{N_{\text{occ}}}=\left\langle\left.\left(\frac{\partial \phi_i}{\partial R_{K\alpha}}\right)_{\vec{R}^0}\right|\left(\frac{\partial \phi_i}{\partial B_\beta}\right)_{\vec{B}=0}\right\rangle
    \end{equation}
    
    ist ein deutlich größerer Aufwand erforderlich. Die $\phi_i$ sind die besetzten \acp{mo}. Wie bereits in Kapitel \ref{theo:nmr} beschrieben, lasst sich die \ac{mo}-Ableitung nach einer Komponente des externen magnetischen Feldes im Rahmen des \ac{cphf}-Formalismus als 
    \begin{equation}\label{eq:vcdcphf}
    \left(\frac{\partial \phi_i}{\partial B_\beta}\right)_{\vec{B}=0}=\phi_i^{B_\beta}=\sum_{\mu=1}^{N_{\text{BF}}}\left[c_{\mu i}\chi_\mu^{B_\beta}+\sum_{p=1}^{N_{\text{MO}}}c_{\mu p}U_{ip}^{B_\beta}\chi_\mu\right]
	\end{equation}
	ausdrücken. Die Koeffizientenmatrix $U_{ip}^{B_\beta}$ beschreibt die Änderung der Molekülorbitale durch die Störung des äußeren Magnetischen Feldes $\vec{B}$. Sie wird durch Lösen der entsprechenden \ac{cphf}-Gleichungen erhalten, ganz analog zur Vorgehensweise bei der Berechnung von \ac{nmr}-Abschirmkonstanten. Durch die Kombination der Gleichungen (\ref{eq:vcdaatel}) und  (\ref{eq:vcdcphf}) wird der zu implementierende Ausdruck für den elektronischen Anteil des \ac{aat} erhalten
	\begin{equation}\label{eq:vcdaatelfinal}
	\begin{aligned}
	I^K_{\alpha\beta}=&\sum_{i=1}^{N_{\text{occ}}}\sum_{\mu,\nu=1}^{N_{\text{BF}}}\left[c_{\mu i}c_{\nu i}\left\langle\chi_\mu^{R_{K\alpha}}\vert\chi_\nu^{B_\beta}\right\rangle+\sum_{p=1}^{N_{\text{MO}}}c_{\mu i}c_{\nu p}U_{ip}^{B_\beta}\left\langle\chi_\mu^{R_{K\alpha}}\vert\chi_\nu\right\rangle\right.\\
	&\left.+\sum_{p=1}^{N_{\text{MO}}}c_{\mu i}c_{\nu p}U_{ip}^{R_{K_\alpha}}\left\langle\chi_\mu\vert\chi_\nu^{B_{\beta}}\right\rangle+\sum_{p,q=1}^{N_{\text{MO}}}c_{\mu p}c_{\nu q}U_{ip}^{R_{K_\alpha}}U_{iq}^{B_\beta}\left\langle\chi_\mu\vert\chi_\nu\right\rangle\right].
	\end{aligned}
	\end{equation}
	Durch die Koeffizientenmatrix $U_{ip}^{R_{K_\alpha}}$ wird die Response der Wellenfunktion auf die Verrückung des Kerns $K$ beschrieben. Analog zu $U_{ip}^{B_\beta}$ werden auch sie durch Lösen der entsprechenden \ac{cphf}-Gleichungen erhalten. Gebraucht werden sie ebenfalls zur Berechnung von Kraftkonstanten, wie sie im \textsc{Turbomole} Modul \texttt{aoforce}\supercite{deglmann2002efficient} berechnet werden.
	\subsection{Implementierung}
	Die in Gleichung (\ref{eq:vcdaatelfinal}) auftretenden Integrale, welche die Ableitung nach dem externen magnetischen Feld enthalten, lassen sich für die $x$-Komponente des $B$-Feldes wie folgt umschreiben
	
	\begin{equation}
	\begin{aligned}
	  \left\langle\chi_\mu\vert\chi_\nu^{B_x}\right\rangle&=\left.\left\langle\chi_\mu\left|\frac{\partial}{\partial B_x}\right.\chi_\nu\right\rangle\right|_{\vec{B}=0}=\left\langle\chi_\mu^{\vec{B}=0}\left|\frac{-\iu}{2c}\left(R_{\nu y}z-R_{\nu z}y\right)\right|\chi_\nu^{\vec{B}=0}\right\rangle\\
	  &=\frac{\iu}{2c}\left(R_{\nu z}\left\langle\chi_\mu^{\vec{B}=0}\left|y\right|\chi_\nu^{\vec{B}=0}\right\rangle-R_{\nu y}\left\langle\chi_\mu^{\vec{B}=0}\left|z\right|\chi_\nu^{\vec{B}=0}\right\rangle\right).
	\end{aligned}
	\end{equation}
	
	Analog ergeben sich die Ableitungen nach der $y$- 
	
		\begin{equation}
	  \left\langle\chi_\mu\vert\chi_\nu^{B_y}\right\rangle=\frac{\iu}{2c}\left(R_{\nu x}\left\langle\chi_\mu^{\vec{B}=0}\left|z\right|\chi_\nu^{\vec{B}=0}\right\rangle-R_{\nu z}\left\langle\chi_\mu^{\vec{B}=0}\left|x\right|\chi_\nu^{\vec{B}=0}\right\rangle\right)
	\end{equation}
	
	und $z$-Komponente
	
		\begin{equation}
	  \left\langle\chi_\mu\vert\chi_\nu^{B_z}\right\rangle=\frac{\iu}{2c}\left(R_{\nu y}\left\langle\chi_\mu^{\vec{B}=0}\left|x\right|\chi_\nu^{\vec{B}=0}\right\rangle-R_{\nu x}\left\langle\chi_\mu^{\vec{B}=0}\left|y\right|\chi_\nu^{\vec{B}=0}\right\rangle\right).
	\end{equation}
	\subsubsection{Symmetrieausnutzung}
	Nur wenige chirale Moleküle besitzen eine Symmetrie. Sie gehören alle zu den Punktgruppen $C_{\textrm{n}}$, $D_{\textrm{n}}$, $T$, $O$ oder $I$ und besitzen damit nur Drehachsen als Symmetrieelemente. Die bereits bestehende Implementierung der Symmetrieausnutzung\supercite{haser1991molecular} für die Module \texttt{mpshift} und \texttt{aoforce} kann auch für die Berechnung von \ac{vcd} Spektren verwendet werden. Dabei ist jedoch zu beachten, dass die Symmetrieausnutzugn im Modul \texttt{mpshift} keine Punktgruppen mit reduziblen $e$-Darstellungen unterstützt, was die Punktgruppen $C_{\textrm{n>2}}$ und $T$ unter den chiralen Punktgruppen betrifft. In diesen Fällen werden die \ac{mo}-Koeffizienten nach $C_1$ transformiert und \texttt{mpshift} wird ohne Symmetrieausnutzung verwendet. Die $U_{pi}^{B_\beta}$ Koeffizienten werden in die \ac{cao}-Basis transformiert, bevor sie für die Weiterverarbeitung im Modul \texttt{aoforce} auf der Festplatte gespeichert werden. Letzteres unterstützt die volle Symmetrieausnuztung, was wichtig ist, in Anbetracht der Tatsache, dass dies den zeitbestimmenden Schritt darstellt. In \texttt{aoforce} erfolgt die Berechnung der $U_{pi}^{R_{K_\alpha}}$, die $U_{pi}^{B_\beta}$ werden von der Festplatte eingelesen und für die volle Symmetrieausnutzung in die \ac{sao}-Basis transformiert. Abschließend erfolgt die Berechnung der \ac{vcd}-Intensitäten. 
	\subsection{GALLIER: Visualisierung von VCD Spektren}
	Zur Visualisierung der berechneten \ac{vcd}-Intensitäten und um einen schnellen, ersten Eindruck vom \ac{vcd}-Spektrum zu verschaffen, wurde das Pythonskript \ac{gallier} erstellt. Die berechneten \ac{vcd}-Intensitäten und die korrespondierenden Frequenzen werden von dem Skript eingelesen und können im Anschluss daran mit Gauß- oder Lorentzförmigen Kurven verbreitert werden, wodurch ein simuliertes Spektrum erhalten wird. Der Benutzer hat dabei die Wahl ob nur die Intensitäten, nur das simulierte Spektrum oder beides visualisiert werden soll. Standardmäßig wird dafür die von \texttt{aoforce} ausgegebene Datei \texttt{vibspectrum} eingelesen, wodurch auch simultan (oder ausschließlich) das Infrarotspektrum des untersuchten Moleküls visualisiert werden kann. Außerdem besteht zusätzlich die Möglichkeit eine beliebige $x,y$-Datei einzulesen um ein beliebiges Spektrum zu simulieren. \ac{gallier} erzeugt dafür ein Eingabeskript für das Visualisierungsprogramm gnuplot\supercite{gnuplot}, sowie eine Datei mit den Rohdaten für jedes beliebige Visualisierungsprogramm. Die Halbwertsbreite sowie der zu zeichnende Bereich des Spektrums können direkt in \ac{gallier} ausgewählt werden.
\section{meta-GGA Funktionale}
	\subsection{Theorie}
	\subsection{Implementierung}
	
	