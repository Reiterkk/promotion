Das Modul \texttt{mpshift} umfasste bisher die Möglichkeit zur Berechnung von chemischen Abschirmungskonstanten in Molekülen ohne schwere Elemente (ab etwa $Z$=36) in der Gasphase. Diese Berechnungen konnten auf Hartree-Fock, \ac{dft} (für \ac{lda}- und \ac{gga}-Funktionale) und \ac{mp2} Nievau durchgeführt werden. Des Weiteren konnte die nach den Komponenten des Magnetfeldes abgeleitete Dichtematrix dem externen Programm \ac{gimic} zur Weiterverarbeitung bereit gestellt werden. Die folgenden Kapitel beschreiben die Erweiterung der Funktionalität des Moduls die im Rahmen dieser Arbeit implementiert wurden. Im Einzelnen sind dies die Berücksichtigung relativistischer Effekte auf Nachbaratome in Molekülen mit schweren Elementen durch relativistische \acp{ecp}, die Einbeziehung von Umgebungseffekten sowie die Bereitstellung der magnetischen Response zur Berechnung von \ac{vcd}-Spektren. Weiterhin wurde das Modul um die Möglichkeit ergänzt, \ac{mgga}-Funktionale für die Berechnung der Abschirmungskonstanten auf \ac{dft}-Niveau zu verwenden. Die eigentliche Implementierung dieses letzten Punktes erfolgte jedoch nicht von mir, sondern von Fabian Mack im Rahmen seiner Masterarbeit.\supercite{mack2017} 
\vfill

\section{Berücksichtigung von Umgebungseffekten}
Isotrope chemische Verschiebungen werden üblicherweise in Lösung gemessen. Je nachdem wie stark das zu untersuchende Molekül mit dem Lösungsmittel wechselwirkt, hat das Lösungsmittel einen mehr oder weniger stark ausgeprägten Einfluss auf das gemessene Spektrum. Das \ac{cosmo}\supercite{klamt1993cosmo}, ein Kontinuumsmodell, ist in der Quantenchemie ein bewährtes Verfahren zur Berücksichtigung von Umgebungseffekten. Neben den Einflüssen des Lösungsmittels lassen sich damit für Ionen auch die Ladungen kompensieren ohne die Gegenionen explizit mitrechnen zu müssen. Insbesondere hoch geladene Anionen lassen sich ohne eine solche Ladungskompensation nur schwer oder gar nicht berechnen. 
	\subsection{Theorie}
	\subsection{Implementierung}
	\subsection{Testrechnungen}
	
\section{Skalar-relativistische Effekte durch effektive Kernpotentiale}
	\subsection{Theorie}
	\subsection{Implementierung}
	
\section{Berechnung von Vibrational Circular Dichroism Spektren}
	\subsection{Theorie}
	Die im Experiment gemessenen Intensitäten der \ac{vcd}-Spektroskopie $I_n$ sind proportional zu den in quantenchemischen Rechnungen zugänglichen Rotationsstärken $R_n$. Letztere werden aus dem Skalaprodukt vom \textcolor{myred}{elektrischen/\-elektronischen} und vom magnetischen Übergangsdipolmoment, $\vec{\mu}_n^{\,\text{el}}$ und $\vec{\mu}_n^{\,\text{mag}}$,  erhalten. Somit ergibt sich die \ac{vcd}-Intensität
	\begin{equation}
	  I_n\approx R_n = Im(\vec{\mu}_n^{\,\text{el}}\cdot\vec{\mu}_n^{\text{\,mag}})
	\end{equation}
	für einen Übergang aus dem Schwingungsgrundzustand in den angeregten Schwingungszustand $n$.\supercite{stephens1985theory,stephens1985vibrational} Im Rahmen der harmonishen Näherung sind das \textcolor{myred}{elektrische/\-elektronische} und das magnetische Übergangsdipolmoment gegeben durch \textcolor{myred}{(Zitat?)}
	
	\begin{equation}
	  (\mu_n^{\text{el}})_\beta=\sqrt{\frac{\hbar}{\omega_n}}\sum_{K\alpha}P_{\alpha\beta}^K S_{K\alpha,n}
	\end{equation}
	\begin{equation}
	  (\mu_n^{\text{mag}})_\beta=-\sqrt{2\hbar^3\omega_n}\sum_{K\alpha}M_{\alpha\beta}^KS_{K\alpha,n}.
	\end{equation}
	Hierbei ist $I$ die Zählvariable für die Atomkerne, $\alpha$ und $\beta$ beschreiben kartesische Koordinaten, $\omega_n$ ist die Schwingungsfrequenz der $n$-ten Schwingung und $S_{I\alpha,n}$ ist die Transformationsmatrix von kartesichen zu Normalkoordinaten. Sowohl der sogenannte \ac{apt} (Gleichung (\ref{eq:apt}) als auch der sogenannte \ac{aat} (Gleichung (\ref{eq:aat})) lassen sich in einen elektronischen und einen Kernbeitrag aufteilen
	\begin{equation}\label{eq:apt}
	  P^K_{\alpha\beta}=E^K_{\alpha\beta}+N^K_{\alpha\beta}
	\end{equation}
	\begin{equation}\label{eq:aat}
   	  M^K_{\alpha\beta}=I^K_{\alpha\beta}+J^K_{\alpha\beta}.
	\end{equation}
	Die Berechnung der Kernbeiträge 
	\begin{equation}
	  N^K_{\alpha\beta}=eZ_K\delta_{\alpha\beta}
	\end{equation}
	\begin{equation}
	  J^K_{\alpha\beta}=\iu\frac{eZ_K}{4\hbar c}\sum_K^{N_K}\varepsilon_{\alpha\beta\gamma}R^0_{K\gamma}
	\end{equation}
	ist trivial. $e$ ist die Elementarladung, $Z_K$ ist die Ladung des Kerns $K$ und $\delta_{\alpha\beta}$ ist das Kroneckerdelta. $c$ ist die Lichtgeschwindigkeit und $\varepsilon_{\alpha\beta\gamma}$ ist der Levi-Civita-Permutationstensor. Die Position des Kerns $K$ ist durch $R^0_{K\gamma}$ gegeben, wobei $\gamma$ für eine der drei kartesischen Raumkoordinaten steht und die hochgestellte $0$ symbolisiert die Auswertung in der Gleichgewichtsgeometrie.   
	
    Zur Berechnung der elektronischen Beiträge 
    \begin{equation}
      E^K_{\alpha\beta}=\left(\sum_{i=1}^{N_{\text{occ}}}\frac{\partial \langle\phi_i\vert r_\beta\vert\phi_i\rangle}{\partial R_{K\alpha}}\right)_{\vec{R}^0}
    \end{equation}
    \begin{equation}\label{eq:vcdaatel}
      I^K_{\alpha\beta}=\sum_{i=1}^{N_{\text{occ}}}=\left\langle\left.\left(\frac{\partial \phi_i}{\partial R_{K\alpha}}\right)_{\vec{R}^0}\right|\left(\frac{\partial \phi_i}{\partial B_\beta}\right)_{\vec{B}=0}\right\rangle
    \end{equation}
    
    ist ein deutlich größerer Aufwand erforderlich. Die $\phi_i$ sind die besetzten \acp{mo}. Wie bereits in Kapitel \ref{theo:nmr} beschrieben, lasst sich die \ac{mo}-Ableitung nach einer Komponente des externen magnetischen Feldes im Rahmen des \ac{cphf}-Formalismus als 
    \begin{equation}\label{eq:vcdcphf}
    \left(\frac{\partial \phi_i}{\partial B_\beta}\right)_{\vec{B}=0}=\phi_i^{B_\beta}=\sum_{\mu=1}^{N_{\text{BF}}}\left[c_{\mu i}\chi_\mu^{B_\beta}+\sum_{p=1}^{N_{\text{MO}}}c_{\mu p}U_{ip}^{B_\beta}\chi_\mu\right]
	\end{equation}
	ausdrücken. Die Koeffizientenmatrix $U_{ip}^{B_\beta}$ beschreibt die Änderung der Molekülorbitale durch die Störung des äußeren Magnetischen Feldes $\vec{B}$. Sie wird durch Lösen der entsprechenden \ac{cphf}-Gleichungen erhalten, ganz analog zur Vorgehensweise bei der Berechnung von \ac{nmr}-Abschirmkonstanten. Durch die Kombination der Gleichungen (\ref{eq:vcdaatel}) und  (\ref{eq:vcdcphf}) wird der zu implementierende Ausdruck für den elektronischen Anteil des \ac{aat} erhalten
	\begin{equation}\label{eq:vcdaatelfinal}
	\begin{aligned}
	I^K_{\alpha\beta}=&\sum_{i=1}^{N_{\text{occ}}}\sum_{\mu,\nu=1}^{N_{\text{BF}}}\left[c_{\mu i}c_{\nu i}\left\langle\chi_\mu^{R_{K\alpha}}\vert\chi_\nu^{B_\beta}\right\rangle+\sum_{p=1}^{N_{\text{MO}}}c_{\mu i}c_{\nu p}U_{ip}^{B_\beta}\left\langle\chi_\mu^{R_{K\alpha}}\vert\chi_\nu\right\rangle\right.\\
	&\left.+\sum_{p=1}^{N_{\text{MO}}}c_{\mu i}c_{\nu p}U_{ip}^{R_{K\alpha}}\left\langle\chi_\mu\vert\chi_\nu^{B_{\beta}}\right\rangle+\sum_{p,q=1}^{N_{\text{MO}}}c_{\mu p}c_{\nu q}U_{ip}^{R_{K\alpha}}U_{iq}^{B_\beta}\left\langle\chi_\mu\vert\chi_\nu\right\rangle\right].
	\end{aligned}
	\end{equation}
	Durch die Koeffizientenmatrix $U_{ip}^{R_{K\alpha}}$ wird die Response der Wellenfunktion auf die Verrückung des Kerns $K$ beschrieben. Analog zu $U_{ip}^{B_\beta}$ werden auch sie durch Lösen der entsprechenden \ac{cphf}-Gleichungen erhalten. Gebraucht werden sie ebenfalls zur Berechnung von Kraftkonstanten, wie sie im \textsf{Turbomole} Modul \texttt{aoforce}\supercite{deglmann2002efficient} berechnet werden.
	\subsection{Implementierung}
	Die in Gleichung (\ref{eq:vcdaatelfinal}) auftretenden Integrale, welche die Ableitung nach dem externen magnetischen Feld enthalten, lassen sich für die $x$-Komponente des $B$-Feldes wie folgt umschreiben
	
	\begin{equation}
	\begin{aligned}
	  \left\langle\chi_\mu\vert\chi_\nu^{B_x}\right\rangle&=\left.\left\langle\chi_\mu\left|\frac{\partial}{\partial B_x}\right.\chi_\nu\right\rangle\right|_{\vec{B}=0}=\left\langle\chi_\mu^{\vec{B}=0}\left|\frac{-\iu}{2c}\left(R_{\nu y}z-R_{\nu z}y\right)\right|\chi_\nu^{\vec{B}=0}\right\rangle\\
	  &=\frac{\iu}{2c}\left(R_{\nu z}\left\langle\chi_\mu^{\vec{B}=0}\left|y\right|\chi_\nu^{\vec{B}=0}\right\rangle-R_{\nu y}\left\langle\chi_\mu^{\vec{B}=0}\left|z\right|\chi_\nu^{\vec{B}=0}\right\rangle\right).
	\end{aligned}
	\end{equation}
	
	Analog ergeben sich die Ableitungen nach der $y$- 
	
		\begin{equation}
	  \left\langle\chi_\mu\vert\chi_\nu^{B_y}\right\rangle=\frac{\iu}{2c}\left(R_{\nu x}\left\langle\chi_\mu^{\vec{B}=0}\left|z\right|\chi_\nu^{\vec{B}=0}\right\rangle-R_{\nu z}\left\langle\chi_\mu^{\vec{B}=0}\left|x\right|\chi_\nu^{\vec{B}=0}\right\rangle\right)
	\end{equation}
	
	und $z$-Komponente
	
		\begin{equation}
	  \left\langle\chi_\mu\vert\chi_\nu^{B_z}\right\rangle=\frac{\iu}{2c}\left(R_{\nu y}\left\langle\chi_\mu^{\vec{B}=0}\left|x\right|\chi_\nu^{\vec{B}=0}\right\rangle-R_{\nu x}\left\langle\chi_\mu^{\vec{B}=0}\left|y\right|\chi_\nu^{\vec{B}=0}\right\rangle\right).
	\end{equation}

\section{meta-GGA Funktionale}
	\subsection{Theorie}
	\subsection{Implementierung}
	
	